\section{Einleitung}
Tijssen2004, Gruber2008, Mansfield1991, Jaffe1989

Der nennenswerte Einfluss von wissenschaftlicher Forschung auf Technologietrends ist allgemein bekannt und in bedeutenden Studien nachgewiesen.Mansfield1991 S. 11, Gruber2008 S. ?, Tegarden2012 S. 599. Bereits die geographische Nähe von Zentren der Spitzentechnologie wie das Silicon Valley oder die Massachusetts Route 128
zu führenden Universitäten lässt Rückschlüsse darauf zu.\footcite[Vgl.][S. 957]{Jaffe1989} Nach einer Studie von Mansfield wären etwa 10~\% der neuen Produkte im Betrachtungszeitraum gar nicht oder nur mit erheblicher Zeitverzögerung entwickelt worden.

Dennoch ist sie (New Economy?) nicht ausschließlich damit begrenzt. Eine nicht zu unterschätzende Rolle (Betamax vs VHS o. USB vs. Firewire) spielen strategische Entscheidungen von Unternehmen (Gruber2008), die aufgrund von Prognosen, Machtkämpfen oder politischen Gegebenheiten durch Entscheidungsträger fallen können.


Um der Wahrheit ein bisschen näher zu kommen, ist der Vergleich von Technologielebenszyklen in Wissenschaft und Wirtschaft eine Möglichkeit, die Entstehung von Trends herauszufinden.