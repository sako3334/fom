\section{Einleitung}
Die wissenschaftliche Forschung wurde als wichtiger Einflussfaktor für die Entstehung neuer Technologien und Technologietrends in bedeutenden Studien nachgewiesen.\footcite[Vgl.][S.~11]{Mansfield1991}\footcite[Vgl.][S.~1652]{Gruber2008}\footcite[Vgl.][S.~599]{Tegarden2012} Nach Jaffe wird diese Erkenntnis bereits durch die geographische Nähe von Zentren der Spitzentechnologie wie das Silicon Valley oder die Massachusetts Route 128 zu führenden Universitäten gestützt.\footcite[Vgl.][S.~967f]{Jaffe1989} Nach einer Studie von Mansfield wären etwa 11~\% aller Produkte einer Auswahl aus sieben Fertigungsindustrien im Betrachtungszeitraum gar nicht oder nur mit erheblicher Zeitverzögerung entwickelt worden, wäre dem nicht eine entsprechende wissenschaftliche Forschung vorausgegangen.\footcite[Vgl.][S.~2]{Mansfield1991}

Dennoch liegt die maßgebliche Entscheidung über den Einsatz sowie die Weiterentwicklung neuer Technologien vorwiegend in Händen von Unternehmern, Managern und sonstigen Entscheidungsträgern der praktizierenden Wirtschaft. Die wiederum treffen ihre Entscheidungen aufgrund einer Vielzahl von Faktoren, mit dem vorrangigen Ziel den Unternehmenserfolg zu steigern. 

Nach Beyer ist der Einfluss auf Technologietrends durch Wirtschaftsmedien höher als durch wissenschaftliche Artikel, da sie von Managern aufgrund des gewohnten Fachjargons sowie der Praxisrelevanz bevorzugt gelesen werden.\footcite[Vgl.][S.~472]{Beyer1992} Barley et al. fanden sogar heraus, dass sich gängige Begriffe der Wirtschaft in wissenschaftlicher Literatur verzögert manifestieren, folglich der Einfluss unidirektional von Unternehmern in Richtung Akademiker stattfindet.\footcite[Vgl.][S.~52]{Barley1988} Nach Spell hängt das allerdings eher damit zusammen, dass wissenschaftliche Artikel einem Peer-Review unterzogen werden, welcher Monate bis Jahre in Anspruch nehmen kann, bis sie in Fachartikeln erscheinen, als dass wissenschaftliche Forschungsschwerpunkte stets aus Wirtschaftsjournalen gespeist würden. \footcite[Vgl.][S.~345]{Spell1999}

\subsection{Problemstellung}
Eine gegenseitige Beeinflussung hinsichtlich der Prognose von Technologietrends zwischen Entscheidungsträgern der Wirtschaft und akademischen Forschern findet somit zweifelsohne statt. Gleichzeitig ist aufgrund teils unterschiedlicher Interessen beider Parteien eine Diskrepanz hierbei zu erwarten.

Es stellt sich die Frage, ob und in welchem Ausmaß sich prognostizierte Technologietrends aus der Wirtschaft in wissenschaftlichen Fachartikel widerspiegeln.


Technologiethemen insbesondere in der IT sind ständiger Veränderung unterworfen

Es stellt sich die Frage welche Trends in naher Vergangehit verfolgt wurden und welche zum aktuellen Zeitpunkt verfolgt werfden.

Vergleich von Technologielebenszyklen in Wissenschaft und Wirtschaft eine Möglichkeit, die Entstehung von Trends herauszufinden.

Da die Zusammenarbeit Synergieeffekte birgt, potential hat, soll diese Arbeit dazu helfen, wo es unterschiedliche Trendauffassungen gibt.

Das sind unter anderem die Marktanforderungen, Prognosen, Machtkämpfe oder politische Gegebenheiten durch Entscheidungsträger

(Betamax vs VHS o. USB vs. Firewire) spielen strategische Entscheidungen von Unternehmen (Gruber2008), die aufgrund von 

\subsection{Zielsetzung}
Beide Strömungen aus wirtschaft und wissenschaft

Ein bekannter Indikator dafür ist der Gartner Hype Cycle, welcher jährlich auf Basis von Interviews mit führenden 

Auufrischen deer Korrelation zwischen trend in wirtschaft und wissenschaft