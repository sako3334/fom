\section{Einleitung}
Die wissenschaftliche Forschung wurde als wichtiger Einflussfaktor für die Entstehung neuer Technologien und Technologietrends in bedeutenden Studien nachgewiesen.\footcite[Vgl.][S.~11]{Mansfield1991}\footcite[Vgl.][S.~1652]{Gruber2008}\footcite[Vgl.][S.~599]{Tegarden2012} Nach Jaffe wird diese Erkenntnis bereits durch die geographische Nähe von Zentren der Spitzentechnologie wie das Silicon Valley oder die Massachusetts Route 128 zu führenden Universitäten gestützt.\footcite[Vgl.][S.~967f]{Jaffe1989} Nach einer Studie von Mansfield wären etwa 11~\% aller Produkte einer Auswahl aus sieben Fertigungsindustrien im Betrachtungszeitraum gar nicht oder nur mit erheblicher Zeitverzögerung entwickelt worden, wäre dem nicht eine entsprechende wissenschaftliche Forschung vorausgegangen.\footcite[Vgl.][S.~2]{Mansfield1991}

Dennoch liegt die maßgebliche Entscheidung über den Einsatz sowie die Weiterentwicklung neuer Technologien vorwiegend in Händen von Unternehmern, Managern und sonstigen Entscheidungsträgern der praktizierenden Wirtschaft. Die wiederum treffen ihre Entscheidungen aufgrund einer Vielzahl von Faktoren, mit dem vorrangigen Ziel den Unternehmenserfolg zu steigern. 

Nach Beyer ist der Einfluss auf Technologietrends durch andere Publikationen höher als durch wissenschaftliche Artikel, da sie von Managern aufgrund des gewohnten Fachjargons sowie der Praxisrelevanz bevorzugt gelesen werden.\footcite[Vgl.][S.]{Beyer1992} Barley et al. fanden sogar heraus, dass sich gängige Begriffe in der wissenschaftlichen Literatur verzögert manifestieren. Dies könne unter anderem damit zusammenhängen, dass diese einem Peer-Review unterzogen werden, welcher Monate bis Jahre in Anspruch nehmen kann.\footcite[Vgl.][S.]{Barley1988}
Eher mechanischer Natur...\footcite[Vgl.][S.345]{Spell1999}

Um der Wahrheit ein bisschen näher zu kommen, ist der Vergleich von Technologielebenszyklen in Wissenschaft und Wirtschaft eine Möglichkeit, die Entstehung von Trends herauszufinden.

Ein bekannter Indikator dafür ist der Gartner Hype Cycle, welcher jährlich auf Basis von Interviews mit führenden 


\subsection{Problemstellung}
Da die Zusammenarbeit Synergieeffekte birgt, potential hat, soll diese Arbeit dazu helfen, wo es unterschiedliche Trendauffassungen gibt.



Das sind unter anderem die Marktanforderungen, Prognosen, Machtkämpfe oder politischen Gegebenheiten durch Entscheidungsträger



(Betamax vs VHS o. USB vs. Firewire) spielen strategische Entscheidungen von Unternehmen (Gruber2008), die aufgrund von 