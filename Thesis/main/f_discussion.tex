\section{Diskussion der Ergebnisse}
Im Anschluss an die empirische Analyse werden die Beobachtungen nun in soweit interpretiert, wie sie der Untersuchung der Leitfragen dienlich sind.

\subsection{Trendverhältnis zwischen Wirtschaft und Wissenschaft}
Im \glqq Hype Cycle\grqq~wird der Zenit aus ca. \numprint{2000} Trendtechnologien abgebildet. Das bedeutet, dass unabhängig von der sichtbaren Position die Technologie in der Wirtschaft eine hohe Popularität genießt. Bei einem weitestgehend übereinstimmenden Trendverständnis in Wirtschaft und Wissenschaft wäre die Konsequenz, dass die Technologien im \glqq Hype Cycle\grqq~ausnahmslos in der wissenschaftlichen Literatur mit entsprechend hohen Publikationen vorhanden sein müssten. Die gesamte Liste der analysierten Technologien ist jedoch nicht öffentlich verfügbar, womit sich die Analyse auf den sichtbaren Teil beschränken muss.

Deshalb wird durch die Suche in der wissenschaftlichen Literatur nach diesen Technologien ein heuristischer Versuch unternommen, eine Korrelation zwischen den beiden Domänen herauszufinden.

In Leitfrage L1 geht es um das Verhältnis der ausgewählten Technologien aus dem \glqq Gartner Hype Cycle\grqq~zu wissenschaftlichen Publikationen. Dazu wurden die Trendverläufe visualisiert und klassifiziert, wovon im Folgenden Gebrauch gemacht wird.

\subsubsection{Vergleich der Literaturdatenbanken}
Die erste Untersuchung gilt den Unterschieden der einzelnen Suchmaschinen und den zugehörigen Veröffentlichungstypen bezüglich der Publikationsmengen im Betrachtungszeitraum. Diese sind aus Tabelle \ref{tab:dist_full_pub} zu entnehmen.

Das \ac{WoS} enthält mit Abstand die meisten Daten über Fachartikel mit steigender Tendenz. Im Vergleich dazu werden dort Konferenzbeiträge am wenigsten veröffentlicht.

Dabei ist es wichtig zu wissen, dass im \ac{WoS} über die Technologien hinaus auch weitere Fachgebiete, wie der Medizin, der Sozialwissenschaften etc. indiziert werden. Das ist eine Erklärung für die Mengendifferenz zum \ac{IEEE}, das ausschließlich auf Technologien im weitesten Sinne spezialisiert ist. Hier wiederum sind Konferenzbeiträge mit einem Verhältnis von ca. 1:4 überrepräsentiert. Im Gegensatz dazu ist die Verteilung im \ac{ACM}, das ausschließlich auf Informationstechnologien spezialisiert ist, in einem ausgewogenen Verhältnis. Es enthält insgesamt die wenigsten Veröffentlichungen.

\subsubsection{Unterschiede in Technologiebegriffen}
Bei der Suche mit unveränderten Technologiebegriffen in den wissenschaftlichen Datenbanken fällt auf, dass ein Großteil der Begriffe in der wissenschaftlichen Literatur kaum vorkommt.

Aus Tabelle \ref{tab:dist_exact} geht hervor, dass lediglich die Technologien \glqq Blockchain\grqq, \glqq Machine Learning\grqq~und \glqq Autonomous Vehicles\grqq~eine gewisse Schwelle an Publikationen überschreiten, die erkennbar macht, dass die Technologien in der Wissenschaft eine gewisse Relevanz besitzen.

Zu den Technologien \glqq Gesture Control Devices\grqq, \glqq Cognitive Expert Advisors\grqq~und \glqq Software-Defined Anything\grqq~existieren keine nennenswerten Publikationen. Dies zeigt, dass zumindest bei der Auswahl der Begriffe keine Beeinflussung seitens der Wissenschaft anzunehmen ist.

Auch die übrigen Technologien sind kumuliert über die einzelnen Ver\-öf\-fent\-lich\-ungs\-typen eher marginal vertreten, wenn sie mit der Gesamtanzahl an Veröffentlichungen in Tabelle \ref{tab:dist_full_pub} in Relation gebracht werden. Bei differenzierter Betrachtung der unterschiedlichen Veröffentlichungstypen ist zu erkennen, dass hierbei vorwiegend Veröffentlichungen als Fachartikel im \ac{WoS} sowie als Konferenzbeiträge im \ac{IEEE} den Unterschied zur Bedeutungslosigkeit ausmachen.

Somit ist eine allumfassende Übereinstimmung der Trendtechnologien auszuschließen. Die ausgewählten Technologien lassen sich nicht pauschal bewerten und müssen differenzierter betrachtet werden.

\subsubsection{Betrachtung der Klassen}
Bei Betrachtung der Technologien aus der in Tabelle \ref{tab:class_tech} definierten Perspektive von Klassen lässt sich ebenfalls kein eindeutiges Muster erkennen. Relevante und irrelevante Technologien in der Wissenschaft sind gleichermaßen über alle Klassen verteilt.

Deshalb wird an dieser Stelle eine niedrigere Abstraktionsebene definiert. Sie unterteilt die Technologien in ihre technischen Kategorien, welche in diesem Fall nicht disjunkt sein müssen. Eine Technologie kann somit in mehreren Klassen enthalten sein.

Die vorliegenden Daten lassen sich wieder in drei technische Klassen untergliedern. Die neue Zuordnung ist Tabelle \ref{tab:class_tech_new} zu entnehmen.

\begin{table}
	\caption{Technische Klassifizierung der Technologien}
	\fontfamily{pcr}\selectfont
	\centering
	\label{tab:class_tech_new}
	\begin{tabularx}{\linewidth}{X|p{2cm}XX}
		Technologie & Hardware & IT-Infrastruktur & Künstliche Intelligenz \\
		\hline
		\acs{GCD} & $\checkmark$ & & \\
		\hline
		\acs{MDC} & & $\checkmark$ & \\
		\hline
		\acs{SR} & $\checkmark$ & & $\checkmark$ \\
		\hline
		\acs{B} & & $\checkmark$ & \\
		\hline
		\acs{CH} & $\checkmark$ & & \\
		\hline
		\acs{CEA} & & & $\checkmark$ \\
		\hline
		\acs{ML} & & & $\checkmark$ \\
		\hline
		\acs{SDS} & & $\checkmark$ & \\
		\hline
		\acs{AV} & $\checkmark$ & & $\checkmark$ \\
		\hline
		\acs{NE} & $\checkmark$ & & \\
		\hline
		\acs{SDx} & & $\checkmark$ & \\
		\hline
	\end{tabularx}
	\caption*{Quelle: Eigene Darstellung}
\end{table}

Aus dieser Perspektive zeigt sich, dass mit Ausnahme von \glqq Blockchain\grqq~die Technologien der Kategorie IT-Infrastruktur weniger Beachtung finden und mit Ausnahme von \glqq Cognitive Expert Advisors\grqq~die Technologien der Kategorie Künstliche Intelligenz die größte Beachtung finden. Bei der Kategorie Hardware divergiert die Trendtendenz.

Eine mögliche Erklärung dafür ergibt sich bei näherer Betrachtung der abweichenden Technologien. Das Thema \glqq Blockchain\grqq~ist beispielsweise durch die Beziehung zur Kryptowährung \glqq Bitcoin\grqq~relativ stark in den Massenmedien präsent. Diese Popularität ist möglicherweise auf die Wissenschaft übergesprungen. Für \glqq Cognitive Expert Advisors\grqq~wiederum gibt es eventuell in der wissenschaftlichen Literatur eine abweichende Benennung.

Deshalb ist die Hinzunahme der Ergebnisse mit erweiterten Suchbegriffen für die Analyse weiterhin unerlässlich.

\subsubsection{Analyse der semantisch erweiterten Suche}
Durch Erweiterung der ursprünglichen Bezeichnungen für die Technologien ist die Anzahl der Treffer teilweise deutlich erhöht worden. Sind bei der exakten Suche nur drei von elf Technologien als in der wissenschaftlichen Literatur relevant festzustellen, so sind nach semantischer Ausweitung der Technologiebegriffe mindestens sechs von elf Technologien als relevant einzustufen. In den Tabellen \ref{tab:dist_wos_art}-\ref{tab:dist_acm_proc} ist zu erkennen, dass die Veröffentlichungszahlen im Betrachtungszeitraum nun überwiegend die Gesamtanzahl der Publikationen bei einer exakten Suche übersteigen.

Die Technologien \glqq Smart Robots\grqq, \glqq Connected Home\grqq~und \glqq Cognitive Expert Advisors\grqq~erreichen hierdurch die Stufe von mehr als 500 Publikationen im Betrachtungszeitraum für alle Veröffentlichungstypen bis auf Konferenzbeiträge im \ac{WoS}. Allerdings wurde im Laufe der Analyse mehrfach deutlich, dass für die vorliegende Datenbasis diese Art der Veröffentlichung so gut wie keine Ergebnisse liefert. Deshalb wird sie bei der quantitativen Beobachtung ignoriert. Wird hierbei dennoch das gegenseitige Verhältnis der wenigen Veröffentlichungen betrachtet, führt dies zu einem äquivalenten Ergebnis und zum gleichen Schluss.

Die Technologie \glqq Nanotube Electronics\grqq~erreicht im \ac{WoS} und \ac{IEEE} ein ähnlich hohes Niveau, steigt im \ac{ACM} allerdings nur mäßig an. \glqq Gesture Control Devices\grqq~erreicht auch bei erweitertem Suchbegriff maximal mittleres Niveau, wobei sie dann im Gegenstück des jeweiligen Veröffentlichungstyps irrelevant bleibt.

Bei den übrigen Technologien \glqq Micro Data Centers\grqq, \glqq Software-Defined Security\grqq~und \glqq Software-Defined Anything\grqq~sind keine Synonyme in der wissenschaftlichen Literatur zu finden. Somit gibt es hier keine nennenswerten Veränderungen hinsichtlich der Veröffentlichungszahlen.

Werden nun die Trendverläufe im \glqq Hype Cycle\grqq~mit denen der wissenschaftlichen Literatur gegenübergestellt, sind zum Teil deutliche Unterschiede zu erkennen. An dem Verlauf der Technologie \glqq Machine Learning\grqq~lässt sich dies verdeutlichen.

Im \glqq Hype Cycle\grqq~erscheint die Technologie im Zeitraum von 2015-2017 unter den stärksten Trends. In der wissenschaftlichen Literatur erreicht sie in allen Datenbanken außer bei Konferenzbeiträgen im \ac{ACM} einen ähnlichen Anstieg zur selben Zeit. Dies kann jedoch vernachlässigt werden, da alle anderen Datenbanken ein umso eindeutigeres Bild liefern. Insofern kann hierfür ein symmetrisches und synchrones Verhältnis angenommen werden, auch wenn in der Wissenschaft die Technologie bereits vor dem Jahr 2015 ein relativ hohes Veröffentlichungsniveau aufwies. Denn das Erscheinen im \glqq Hype Cycle\grqq~erfordert die Überschreitung einer Schwelle von Erwartungen bei den Unternehmern. Eine einseitige Beeinflussung ist nicht zu erkennen.

Doch so Ideal das Verhältnis für \glqq Machine Learning\grqq~auch sein mag, verzerrt es umso mehr die Relationen der übrigen Technologien. Auch die als relevant bezeichneten Technologien kommen mit einer Veröffentlichungsmenge von im Schnitt 10~\% da nicht heran. Bei näherer Betrachtung des Kurvenverlaufes in Abbildung \ref{fig:ghc_trend} müssten anderenfalls die Veröffentlichungszahlen im Jahre 2016 von \glqq Autonomous Vehicles\grqq, \glqq Cognitive Expert Advisors\grqq~und \glqq Software-Defined Security\grqq~ein vergleichbares Niveau besitzen, was nicht annähernd der Fall ist.

Um dennoch eine Aussage zum Verhältnis der jeweiligen Trends zu machen, werden die zuvor klassifizierten Technologien nochmal genauer beleuchtet.

Die in der wissenschaftlichen Literatur unbedeutenden Technologien sind allesamt Themen der IT-Infrastruktur, bei denen es sich um sehr spezielle Themen handelt, die wiederum einen ausgewählten Kreis an Interessengruppen überwiegend im Entreprise-Sektor betreffen. Im Gegensatz dazu trifft die in der gleichen Kategorie enthaltene \glqq Blockchain-Technologie\grqq~auf breitere Resonanz in der Gesellschaft, was scheinbar zu einer Steigerung des Forschungsinteresses führt. Allerdings nicht so hoch, dass von einem starken Trend gesprochen werden kann.

Bei Technologien, wie \glqq Smart Robots\grqq, \glqq Cognitive Expert Advisors\grqq, \glqq Machine Learning\grqq~sowie \glqq Autonomous Vehicles\grqq, die in gewissem Maße als Trendschwerpunkt auffallen, handelt es sich eher um gesamtgesellschaftliche Belange, die weitreichende Problemlösungspotentiale involvieren.

Somit ist zum Trendverhältnis festzuhalten, dass das Thema der künstlichen Intelligenz eine hohe Beachtung in der Wissenschaft findet. Diese geht allerdings über den Betrachtungszeitraum hinaus und bedeutet damit nicht, dass die wissenschaftliche Forschung einseitig beeinflusst wird. Die Themen der IT-Infrastruktur finden mit einer Ausnahme keine bemerkenswerte Beachtung. Die Themen zur Hardware bzw. Elektrotechnik scheinen in vorliegendem Fall auch eher eine Nische zu bedienen.

\subsection{Richtung der Trendbeeinflussung}
Für die Beantwortung der anderen beiden Leitfragen, L2 und L3, sind erneut die Kurvenverläufe der Veröffentlichungen zu untersuchen. Dabei geht es darum, Auffälligkeiten bei wissenschaftlichen Publikationen zu beobachten, die bei Ersterscheinung sowie Wegfall im \glqq Hype Cycle\grqq~potentiell auftreten.

Die Leitfrage L2 betrifft hierbei alle elf und L3 nur vier Technologien. Die entsprechenden Jahre sind den Tabellen \ref{tab:ghc_init} und \ref{tab:ghc_abscence} zu entnehmen.

Für die Technologie \glqq Autonomous Vehicles\grqq~können beide Fragen in einem Zuge betrachtet werden. Die Ersterscheinung im \glqq Hype Cycle\grqq~erfolgte im Jahre 2010, um im Jahr darauf wieder herauszufallen. In der dazugehörigen Kurve in Abbildung \ref{fig:av_pub} heben sich die zwei oberen Kurven vom Rest ab. Sowohl im Erscheinungsjahr sowie im Folgejahr sind keine signifikanten Ausschläge zu verzeichnen. Der Anstieg beginnt erst in 2014. Es kann somit keine Korrelation zwischen den Ereignissen im \glqq Hype Cycle\grqq~und den Publikationen in der Wissenschaft beobachtet werden.

Die Technologien \glqq Connected Home\grqq~und \glqq Smart Robots\grqq~erschienen das erste Mal im Jahre 2014 im \glqq Hype Cycle\grqq. In 2015 ist bei drei von sechs Veröffentlichungstypen ein Anstieg zu diesem Zeitpunkt zu erkennen. Bei \glqq Smart Robots\grqq~allerdings eher linear als abrupt.

Auffälligkeiten sind einzig bei Technologien mit einer geringen Veröffentlichungszahl bemerkbar. Bei \glqq Gesture Control Devices\grqq~und \glqq Micro Data Centers\grqq~ist bspw. zu sehen, dass ein Anstieg ab der ersten Veröffentlichung im \glqq Hype Cycle\grqq~festzustellen ist. Dies könnte ein Hinweis darauf sein, dass eine Beeinflussung in äußerst geringem Maße von der Wirtschaft in Richtung Wissenschaft stattfindet, diese aber vernachlässigbar klein ist. Es ist auch denkbar, dass es sich dabei um Kollaborationen zwischen der Wirtschaft und Wissenschaft handelt, wo die Benennung abgestimmt wurde.

Genauso bemerkenswert ist die Tatsache, dass bspw. die Technologie \glqq Cognitive Expert Advisors\grqq~in der Wissenschaft bereits in den 1980er Jahren einen Hype erlebt hat, der jetzt erst in die Wirtschaft eintritt. Es sind daher nahezu beliebig viele Konstellationen des Trendverlaufes auf beiden Seiten vorstellbar und möglich.

Angesichts dessen, ist eine eindeutige und erhöhte Korrelation zwischen der Wirtschaft und Wissenschaft auch auf Basis dieser Leitfragen nicht erkennbar.

\subsection{Kritische Reflexion}
Obwohl die Analyse recht eindeutig ergab, dass zwischen dem \glqq Gartner Hype Cycle\grqq~und der wissenschaftlichen Literatur Diskrepanzen bestehen, basiert die Erkenntnis auf einer begrenzten Anzahl an Technologien. Das bedeutet, dass die Gültigkeit der Auswertung auf die Stichprobe begrenzt ist. Hinzu kommt, dass die internen Ermittlungsmethoden sowie Analysedaten von \glqq Gartner\grqq~nicht ausreichend bekannt sind, was eine mathematische Korrelationsanalyse undurchführbar macht. Die Ausweitung der Daten auf Analysen weiterer Marktforschungsunternehmen wäre demnach zu empfehlen.

Eine weitere Möglichkeit die Zuverlässigkeit der Analyse zu erhöhen, wäre durch die Verschiebung des Betrachtungszeitraumes in die Vergangenheit gegeben. Durch diese Art von Gegenprobe kann die Allgemeingültigkeit der Auswertung verifiziert werden sowie saisonale Unterschiede aufzeigen.

Eine weitere Möglichkeit zur Verfeinerung der Trendanalyse wäre durch Einsatz von automatisierten Verfahren einhergehend mit dem Zugriff auf die \ac{API} der jeweiligen Literaturdatenbank gegeben. Der umgekehrte Vergleich auf Basis der akademischen Technologietrends würde zu den Ursachen in Sachen Diskrepanzen möglicherweise zusätzliche Erkenntnisse liefern.