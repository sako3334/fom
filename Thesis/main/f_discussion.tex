\section{Diskussion der Ergebnisse}
Im Anschluss an die empirische Analyse werden die Beobachtungen nun in soweit interpretiert, wie sie der Untersuchung der Leitfragen dienlich sind.

\subsection{Trendverhältnis zwischen Wirtschaft und Wissenschaft}
Im \glqq Hype Cycle\grqq~wird der Zenit aus ca. \numprint{2000} Trendtechnologien abgebildet. Das bedeutet, dass unabhängig von der sichtbaren Position die Technologie in der Wirtschaft eine hohe Popularität genießt. Bei einem weitestgehend übereinstimmenden Trendverständnis in Wirtschaft und Wissenschaft wäre die Konsequenz, dass die Technologien im \glqq Hype Cycle\grqq~ausnahmslos in der wissenschaftlichen Literatur mit entsprechend hohen Publikationen vorhanden sein müssten. Die gesamte Liste der analysierten Technologien ist jedoch nicht öffentlich verfügbar, womit sich die Analyse auf den sichtbaren Teil beschränken muss.

Deshalb wird durch die Suche in der wissenschaftlichen Literatur nach diesen Technologien ein heuristischer Versuch unternommen, eine Korrelation zwischen den beiden Domänen herauszufinden.

In Leitfrage L1 geht es um das Verhältnis der ausgewählten Technologien aus dem \glqq Gartner Hype Cycle\grqq~zu wissenschaftlichen Publikationen. Dazu wurden die Trendverläufe visualisiert und klassifiziert, wovon im Folgenden Gebrauch gemacht wird.

\subsubsection{Vergleich der Literaturdatenbanken}
Die erste Untersuchung gilt den Unterschieden der einzelnen Suchmaschinen und den zugehörigen Veröffentlichungstypen bezüglich der Publikationsmengen im Betrachtungszeitraum. Diese sind aus Tabelle \ref{tab:dist_full_pub} zu entnehmen.

Das \ac{WoS} enthält mit Abstand die meisten Daten über Fachartikel mit steigender Tendenz. Im Vergleich dazu werden dort Konferenzbeiträge am wenigsten veröffentlicht.

Dabei ist es wichtig zu wissen, dass im \ac{WoS} über die Technologien hinaus auch weitere Fachgebiete, wie der Medizin, der Sozialwissenschaften etc. indiziert werden. Das ist eine Erklärung für die Mengendifferenz zum \ac{IEEE}, das ausschließlich auf Technologien im weitesten Sinne spezialisiert ist. Hier wiederum sind Konferenzbeiträge mit einem Verhältnis von ca. 1:4 überrepräsentiert. Im Gegensatz dazu ist die Verteilung im \ac{ACM}, das ausschließlich auf Informationstechnologien spezialisiert ist, in einem ausgewogenen Verhältnis. Es enthält insgesamt die wenigsten Veröffentlichungen.



\subsubsection{Unterschiede in Technologiebegriffen}
Bei der Suche mit unveränderten Technologiebegriffen in den wissenschaftlichen Datenbanken fällt auf, dass ein Großteil der Begriffe in der wissenschaftlichen Literatur kaum vorkommt.

Aus Tabelle \ref{tab:dist_exact} geht hervor, dass lediglich die Technologien \glqq Blockchain\grqq, \glqq Machine Learning\grqq~und \glqq Autonomous Vehicles\grqq~eine gewisse Schwelle an Publikationen überschreiten, die erkennbar macht, dass die Technologien in der Wissenschaft eine gewisse Relevanz besitzen.

Zu den Technologien \glqq Gesture Control Devices\grqq, \glqq Cognitive Expert Advisors\grqq~und \glqq Software-Defined Anything\grqq~existieren keine nennenswerten Publikationen. Dies zeigt, dass zumindest bei der Auswahl der Begriffe keine Beeinflussung seitens der Wissenschaft anzunehmen ist.

Auch die übrigen Technologien sind kumuliert über die einzelnen Ver\-öf\-fent\-lich\-ungs\-typen eher marginal vertreten, wenn sie mit der Gesamtanzahl an Veröffentlichungen in Tabelle \ref{tab:dist_full_pub} in Relation gebracht werden. Bei differenzierter Betrachtung der unterschiedlichen Veröffentlichungstypen ist zu erkennen, dass hierbei vorwiegend Veröffentlichungen als Fachartikel im \ac{WoS} sowie als Konferenzbeiträge im \ac{IEEE} den Unterschied zur Bedeutungslosigkeit ausmachen.

Somit ist eine allumfassende Übereinstimmung der Trendtechnologien auszuschließen. Die ausgewählten Technologien lassen sich nicht pauschal bewerten und müssen differenzierter betrachtet werden.

\subsubsection{Betrachtung der Klassen}
Bei Betrachtung der Technologien aus der in Tabelle \ref{tab:class_tech} definierten Perspektive von Klassen lässt sich ebenfalls kein eindeutiges Muster erkennen. Relevante und irrelevante Technologien in der Wissenschaft sind gleichermaßen über alle Klassen verteilt.

Deshalb wird an dieser Stelle eine niedrigere Abstraktionsebene definiert. Sie unterteilt die Technologien in ihre technischen Kategorien, welche in diesem Fall nicht disjunkt sein müssen. Eine Technologie kann somit in mehreren Klassen enthalten sein.

Die vorliegenden Daten lassen sich wieder in drei technische Klassen untergliedern. Die neue Zuordnung ist Tabelle \ref{tab:class_tech_new} zu entnehmen.

\begin{table}
	\caption{Technisches Klassifizierung der Technologien}
	\fontfamily{pcr}\selectfont
	\centering
	\label{tab:class_tech_new}
	\begin{tabularx}{\linewidth}{X|p{2cm}XX}
		Technologie & Hardware & IT-Infrastruktur & Künstliche Intelligenz \\
		\hline
		\acs{GCD} & $\checkmark$ & & \\
		\hline
		\acs{MDC} & & $\checkmark$ & \\
		\hline
		\acs{SR} & $\checkmark$ & & $\checkmark$ \\
		\hline
		\acs{B} & & $\checkmark$ & \\
		\hline
		\acs{CH} & $\checkmark$ & & \\
		\hline
		\acs{CEA} & & & $\checkmark$ \\
		\hline
		\acs{ML} & & & $\checkmark$ \\
		\hline
		\acs{SDS} & & $\checkmark$ & \\
		\hline
		\acs{AV} & $\checkmark$ & & $\checkmark$ \\
		\hline
		\acs{NE} & $\checkmark$ & & \\
		\hline
		\acs{SDx} & & $\checkmark$ & \\
		\hline
	\end{tabularx}
	\caption*{Quelle: Eigene Darstellung}
\end{table}

Aus dieser Perspektive zeigt sich, dass mit Ausnahme von \glqq Blockchain\grqq~die Technologien der Kategorie IT-Infrastruktur weniger Beachtung finden und mit Ausnahme von \glqq Cognitive Expert Advisors\grqq~die Technologien der Kategorie Künstliche Intelligenz die größte Beachtung finden. Bei der Kategorie Hardware divergiert die Trendtendenz.

Eine mögliche Erklärung dafür ergibt sich bei näherer Betrachtung der abweichenden Technologien. Das Thema \glqq Blockchain\grqq~ist beispielsweise durch die Beziehung zur Kryptowährung \glqq Bitcoin\grqq~relativ stark in den Massenmedien präsent. Diese Popularität ist möglicherweise auf die Wissenschaft übergesprungen. Für \glqq Cognitive Expert Advisors\grqq~wiederum gibt es eventuell in der wissenschaftlichen Literatur eine abweichende Benennung.

Deshalb ist die Hinzunahme der Ergebnisse mit erweiterten Suchbegriffen für die Analyse weiterhin unerlässlich.

\subsubsection{Analyse der semantisch erweiterten Suche}
Durch Erweiterung der ursprünglichen Bezeichnungen für die Technologien ist die Anzahl der Treffer teilweise deutlich erhöht worden. Sind bei der exakten Suche nur drei von elf Technologien als in der wissenschaftlichen Literatur relevant festzustellen, so sind nach semantischer Ausweitung der Technologiebegriffe mindestens sechs von elf Technologien als relevant einzustufen. In den Tabellen \ref{tab:dist_wos_art}-\ref{tab:dist_acm_proc} ist zu erkennen, dass die Veröffentlichungszahlen im Betrachtungszeitraum nun überwiegend die Gesamtanzahl der Publikationen bei einer exakten Suche übersteigen.

Die Technologien \glqq Smart Robots\grqq, \glqq Connected Home\grqq~und \glqq Cognitive Expert Advisors\grqq~erreichen hierdurch die Stufe von mehr als 500 Publikationen im Betrachtungszeitraum für alle Veröffentlichungstypen bis auf Konferenzbeiträge im \ac{WoS}. Allerdings wurde im Laufe der Analyse mehrfach deutlich, dass für die vorliegende Datenbasis diese Art der Veröffentlichung so gut wie keine Ergebnisse liefert. Deshalb wird sie bei der quantitativen Beobachtung ignoriert. Wird hierbei dennoch das gegenseitige Verhältnis der wenigen Veröffentlichungen betrachtet, führt dies zu einem äquivalenten Ergebnis und zum gleichen Schluss.

Die Technologie \glqq Nanotube Electronics\grqq~erreicht im \ac{WoS} und \ac{IEEE} ein ähnlich hohes Niveau, steigt im \ac{ACM} allerdings nur mäßig an. \glqq Gesture Control Devices\grqq~erreicht auch bei erweitertem Suchbegriff maximal mittleres Niveau, wobei sie dann im Gegenstück des jeweiligen Veröffentlichungstyps irrelevant bleibt.

Bei den übrigen Technologien \glqq Micro Data Centers\grqq, \glqq Software-Defined Security\grqq~und \glqq Software-Defined Anything\grqq~sind keine Synonyme in der wissenschaftlichen Literatur zu finden. Somit gibt es hier keine nennenswerten Veränderungen hinsichtlich der Veröffentlichungszahlen.

Werden nun die Trendverläufe im \glqq Hype Cycle\grqq~mit denen der wissenschaftlichen Literatur gegenübergestellt, sind zum Teil deutliche Unterschiede zu erkennen. An dem Verlauf der Technologie \glqq Machine Learning\grqq~lässt sich dies verdeutlichen.

Im \glqq Hype Cycle\grqq~erscheint die Technologie im Zeitraum von 2015-2017 unter den stärksten Trends. In der wissenschaftlichen Literatur erreicht sie in allen Datenbanken außer bei Konferenzbeiträgen im \ac{ACM} einen ähnlichen Anstieg zur selben Zeit. Dies kann jedoch vernachlässigt werden, da alle anderen Datenbanken ein umso eindeutigeres Bild liefern. Insofern kann hierfür ein symmetrisches und synchrones Verhältnis angenommen werden, auch wenn in der Wissenschaft die Technologie bereits vor dem Jahr 2015 ein relativ hohes Veröffentlichungsniveau aufwies. Denn das Erscheinen im \glqq Hype Cycle\grqq~erfordert die Überschreitung einer Schwelle von Erwartungen bei den Unternehmern. Eine einseitige Beeinflussung ist nicht zu erkennen.

Doch so Ideal das Verhältnis für \glqq Machine Learning\grqq~auch sein mag, verzerrt es umso mehr die Relationen der übrigen Technologien. Auch die als relevant bezeichneten Technologien kommen mit einer Veröffentlichungsmenge von im Schnitt 10~\% da nicht heran. Bei näherer Betrachtung des Kurvenverlaufes in Abbildung \ref{fig:ghc_trend} müssten anderenfalls die Veröffentlichungszahlen im Jahre 2016 von \glqq Autonomous Vehicles\grqq, \glqq Cognitive Expert Advisors\grqq~und \glqq Software-Defined Security\grqq~ein vergleichbares Niveau besitzen, was nicht annähernd der Fall ist.

Um dennoch eine Aussage zum Verhältnis der jeweiligen Trends zu machen, werden die zuvor klassifizierten Technologien nochmal genauer beleuchtet.

Die in der wissenschaftlichen Literatur unbedeutenden Technologien sind allesamt Themen der IT-Infrastruktur, bei denen es sich um sehr spezielle Themen handelt, die wiederum einen ausgewählten Kreis an Interessengruppen überwiegend im Entreprise-Sektor betreffen. Im Gegensatz dazu trifft die in der gleichen Kategorie enthaltene \glqq Blockchain-Technologie\grqq~auf breitere Resonanz in der Gesellschaft, was scheinbar zu einer Steigerung des Forschungsinteresses führt. Allerdings nicht so hoch, dass von einem starken Trend gesprochen werden kann.

Bei Technologien, wie \glqq Smart Robots\grqq, \glqq Cognitive Expert Advisors\grqq, \glqq Machine Learning\grqq~sowie \glqq Autonomous Vehicles\grqq, die in gewissem Maße als Trendschwerpunkt auffallen, handelt es sich eher um gesamtgesellschaftliche Belange, die weitreichende Problemlösungspotentiale involvieren.

\subsection{Richtung der Trendbeeinflussung}

im Verhältnis gegenüberstellen: schwierig, da GHC spitze des Eisbergs zeigt und Pub der wissenschaft meist dartunte liegen außer ML
aus 2000 Technologien durch Gartner im GHC zusamengestellt
Diskrepanzen Schwerpunktsetzung
unterschiedliche Interessen
Unterschiede zwischen den Publikationstypen
Diskussion:
Exakte Suche unzureichend, Unterschiede in Benennung, Gartner fasst evtl. die unterschiedlichen Benennungen der Unternehmen zusammen

Absolute Veröffentlichungszahlen vergleichen: ML viel höher als Rest


Unterschiede je Suchmaschine und Veröffentlichungstyp
Technologien: IEEE Konferenzen, WOS Artikel

Eine einseitige Beeinflussung generell nicht erkennbar

WOS mehrere Fachgebiete inkl. Medizin, Sozialwissenschaften etc., Eingrenzung durch Suchparameter möglich
WoS KB fast immer leer
IEEE und ACM technologieorientierte Spezialdatenbanken, bieten keine Eingrenzung in Themengebiete an

AV hohe Komplexität, lange Dauer bis zur Produktivität <- attraktiv für die Forschung

SR, CH

ML generischer Begriff, spezielle Technologien in Wissenschaft nicht stark vertreten, eher theoretische Modelle

Obwohl Suchbegriff erweitert, GCD wenig Publikationen

In diskussion eine tabelle mit gesamt-pub, trendsträke ghc, Q wissenschaft kumuliert gegenüberstellen

Exakte Suche mit erweiterter suche berücksichtigen

angesichts... :)

Andere Bezeichnung in der Forschung? -> Hinweis auf Diskrepanzen
Theoretische vs. praktische Technologien. Theoretische Idee hinter einer kommerziellen Technologie herausfinden
CEA: machine learning, artificial intelligence

\subsection{Kritische Reflexion}
mehr technologien
mehr einblick in gartner oder weitere trendanalysten der wirtscaft
weitere suchmaschinen einbinden: scopus
andersherum suchen: erst wissenschaftsthemen, methodik entwickeln