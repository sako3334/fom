\section{Ergebnisse der Untersuchung}
Im nächsten Schritt gilt es, einen Eindruck über die Verteilung der Daten im Hinblick auf die Beantwortung der Leitfragen zu gewinnen. Dazu werden sie zunächst einmal adäquat gruppiert und in einem Graphen geplottet.

\subsection{Trendverlauf im \glqq Hype Cyle\grqq}
Die Verteilung der Technologien aus dem Jahre 2016 über alle Veröffentlichungen ist in Abbildung \ref{fig:ghc_trend} zu sehen.

\begin{figure}
	\centering
	\caption{Trendverlauf der Technologien aus dem \glqq Gartner Hype Cycle\grqq}
	\includegraphics[width=\linewidth,clip,trim=0cm 0.5cm 1cm 2cm]{pdf/ghc_trend.pdf}
	\caption*{Quelle: Eigene Darstellung}
	\label{fig:ghc_trend}
\end{figure}

Die Legende listet die Symbole und dazugehörigen Farben zu den abgekürzten Technologien auf, mit denen die Koordinatenpunkte markiert sind. Insgesamt sind elf Technologien über den Zeitraum von 2010 bis 2017 abgebildet. Die Reihenfolge der Auflistung ergibt sich aus der Reihung der ursprünglichen Ausgabe.

Zunächst einmal fällt auf, dass sich die Technologien des ausgewählten Zeitraumes relativ eng um das Jahr der Vergleichsbasis befinden. Die Standardabweichung beträgt aufgerundet gerade einmal 1,8 um das Veröffentlichungsjahr 2016.

Einzig die Trendtechnologie \glqq Autonomous Verhicles\grqq~erstreckt sich, mit einer Unterbrechung im Jahre 2011, über den gesamten Zeitraum. Im Jahre 2014 steigt sie aus der Phase \glqq Innovation Trigger\grqq~in die Phase \glqq Peak of Inflated Expectations\grqq~auf und ist seitdem dort vertreten. Im Jahre 2015 erreicht sie ihre Höchstform als stärkster Trend des Jahres.

Die Technologien \glqq Smart Robots\grqq~und \glqq Connected Home\grqq~erscheinen erstmalig im Jahre 2014 im \glqq Innovation Trigger\grqq~und sind seit 2016 in den \glqq Peak of Inflated Expectations\grqq~aufgestiegen.

Im Jahr darauf erscheinen zwei der Technologien, \glqq Micro Data Centers\grqq~und \glqq Machine Learning\grqq~initial im \glqq Peak of Inflated Expectations\grqq~sowie \glqq Software-Defined Security\grqq~in der Phase \glqq Innovation Trigger\grqq. Die Technologie \glqq Micro Data Centers\grqq~verbleibt im Folgejahr auf vergleichbarem Niveau und fällt im Jahre 2017 aus dem \glqq Hype Cycle\grqq~heraus. \glqq Machine Learning\grqq~hingegen verbleibt bis zur letzten Ausgabe in der gleichen Phase auf höchstem Trendniveau, im Jahre 2016 erreicht sie sogar den höchsten Punkt der Trendkurve. \glqq Software-Defined Security\grqq~erfährt wiederum eine relativ schnelle Reife\-entwicklung. Im Jahre 2016 ist sie bereits im \glqq Peak of Inflated Expectations\grqq~und wieder ein Jahr später im \glqq Trough of Disillusionment\grqq. Damit ist sie die einzige Technologie aus der Datenbasis die alle drei Phasen unmittelbar aufeinanderfolgend belegt.

Da es sich um die Datenbasis handelt, sind alle übrigen Technologien im Jahr 2016 erstmalig in Erscheinung getreten und befinden sich im Abschnitt \glqq Peak of Inflated Expectations\grqq. \glqq Gesture Control Devices\grqq~und \glqq Software-Defined Anything\grqq~sind bereits im darauffolgenden Jahr nicht mehr im \glqq Hype Cycle\grqq. \glqq Blockchain\grqq~und \glqq Nanotube Electronics\grqq~hingegen befinden sich auch in der aktuellsten Ausgabe im \glqq Peak of Inflated Expectations\grqq~und \glqq Cognitive Expert Advisors\grqq~schließlich ist zum gleichen Zeitpunkt eine Ebene höher gekommen, in die Phase \glqq Trough of Disillusionment\grqq.

Für die Leitfrage L2 ist das Jahr der ersten Veröffentlichung einer Technologie im \glqq Hype Cycle\grqq~von Bedeutung. Diese werden in Tabelle \ref{tab:ghc_init} festgehalten.

\begin{table}
	\caption{Erstmaliges Erscheinungsjahr der Technologien im \glqq Gartner Hype Cycle\grqq}
	\fontfamily{pcr}\selectfont
	\centering
	\label{tab:ghc_init}
	\begin{tabularx}{\linewidth}{X|p{3em}XXX}
	Erst\-erscheinung & 2010 & 2014 & 2015 & 2016 \\
	\hline
	Technologien & \acs{AV} & \acs{SR}, \acs{CH} & \acs{MDC}, \acs{ML}, \acs{SDS} & \acs{GCD}, \acs{B}, \acs{CEA}, \acs{NE}, \acs{SDx} \\
	\end{tabularx}
	\caption*{Quelle: Eigene Darstellung}
\end{table}

Für die Beantwortung der Leitfrage L3 werden die Technologien benötigt, die nach Erscheinen in einer Ausgabe des \glqq Hype Cycle\grqq~im Folgejahr weggefallen sind. Die Technologien und das Jahr der Absenz werden deshalb in Tabelle \ref{tab:ghc_abscence} zusammengefasst.

\begin{table}
	\caption{Veröffentlichungsjahre der Absenz von Technologien im \glqq Gartner Hype Cycle\grqq}
	\fontfamily{pcr}\selectfont
	\centering
	\label{tab:ghc_abscence}
	\begin{tabularx}{\linewidth}{X|XX}
		Absenzjahr & 2011 & 2017  \\
		\hline
		Technologien & \acs{AV} & \acs{GCD}, \acs{MDC}, \acs{SDx} \\
	\end{tabularx}
	\caption*{Quelle: Eigene Darstellung}
\end{table}

\subsection{Trendverlauf in der akademischen Forschung}
Die Bestimmung des Trendverlaufes in der akademischen Forschung ist umfangreicher als im \glqq Hype Cycle\grqq. Dies basiert auf der Möglichkeit, die wissenschaftlichen Publikationen im Hinblick auf den Trendverlauf genauer auswerten zu können, da die auswertbaren Parameter breiter gefächert sind.

Zunächst einmal werden unterschiedliche Datenquellen für die Ermittlung der Trend\-stär\-ke eingesetzt, was im Ge\-gen\-satz zum \glqq Hype Cycle\grqq~eine Er\-hö\-hung der Datenmenge herbeiführt. Auch die Unterteilung in unterschiedliche Veröffentlichungstypen erhöht die notwendigen Aktionen für das Visualisieren und Auswerten. Auf der anderen Seite ist durch die Möglichkeit, konkrete Veröffentlichungsmengen zu bestimmen, eine genauere Analyse möglich und nötig.

Wegen der Unterteilung in verschiedene Veröffentlichungstypen, ist sowohl der differenzierte, relative Trend, wie die kumulierte Gesamtmenge der Publikationen zu betrachten.

Im Folgenden wird die Mengenverteilung der Veröffentlichungen zu den einzelnen Technologien betrachtet. Pro Suchmaschine und Veröffentlichungstyp ist jeweils eine Kurve abgebildet. Dies wird durch unterschiedliche Linientypen und Punktsymbole kenntlich gemacht.

In Abbildung \ref{fig:gcd_pub} ist die Verteilung der Veröffentlichungen zur Technologie \glqq Gesture Control Devices\grqq~dargestellt. Es ist festzustellen, dass die absoluten Mengen an Publikationen im Verhältnis zu den Gesamtveröffentlichungen in Tabelle \ref{tab:dist_full_pub} trotz Erweiterung des Suchbegriffes mit weit unter 0,1~\% sehr gering Ausfallen. Der relative Trend zeigt eine Ansammlung bei steigender Tendenz im Jahre 2016 für vier von sechs Kurven. Konferenzbeiträge im \ac{ACM} kommen am häufigsten vor und haben ihren Höhepunkt in den Jahren 2013 und 2014 mit rund doppelt so viel Publikationen wie Konferenzbeiträge bei \ac{IEEE}. Konferenzbeiträge im \ac{WoS} sowie Artikel im \ac{IEEE} kommen vernachlässigbar wenig vor.

\begin{figure}
	\centering
	\caption{Verteilung von Publikationen zu \glqq Gesture Control Devices\grqq}
	\includegraphics[width=\linewidth,clip,trim=0cm 0.5cm 1cm 0.8cm,height=\textheight/3]{pdf/GCD.pdf}
	\caption*{Quelle: Eigene Darstellung}
	\label{fig:gcd_pub}
\end{figure}

Abbildung \ref{fig:mdc_pub} zeigt die entsprechende Graphik zu \glqq \acl{MDC}\grqq. Der Suchbegriff wurde nur um verschiedene Schreibweisen der Technologie erweitert. Die absoluten Häufigkeiten sind noch geringer als die der vorhergehenden Technologie. Konferenzberichte im \ac{WoS} sind nicht vorhanden. Die ersten Publikationen überhaupt sind im Jahre 2011 als Konferenzberichte im \ac{ACM} erschienen. Der relative Trend hingegen liegt auch im Jahre 2016.

\begin{figure}
	\centering
	\caption{Verteilung von Publikationen zu \glqq \acl{MDC}\grqq}
	\includegraphics[width=\linewidth,clip,trim=0cm 0.5cm 1cm 0.8cm,height=\textheight/3]{pdf/MDC.pdf}
	\caption*{Quelle: Eigene Darstellung}
	\label{fig:mdc_pub}
\end{figure}

Der Verlauf von \glqq Smart Robots\grqq~ist in Abbildung \ref{fig:sr_pub} zu sehen. Der Begriff wurde um die unterschiedlichen Bezeichnungen in der Literatur ergänzt. Die Technologie ist im gesamten Betrachtungszeitraum und darüber hinaus in der wissenschaftlichen Literatur vertreten. Eine signifikante Anzahl an Publikationen ist in Konferenzbeiträgen im \ac{IEEE} zu verzeichnen, die mit \numprint{1320} Publikationen ihren Höhepunkt erneut im Jahre 2016 erreichen. Auch wenn sie mit knapp 300 publizierten Fachartikeln deutlich darunter liegen, ist diese Menge bereits größer als zu den bisherigen Technologien. Von 2014 bis 2017 ist die Anzahl an Publikationen stets höher als im Vorjahr. Die übrigen Veröffentlichungen sind im Verhältnis dazu relativ niedrig.

Alles in allem lässt sich sagen, dass die Technologie \glqq Smart Robots\grqq~ein erhöhtes Maß an Aufmerksamkeit in der Wissenschaft im Vergleich zu den bisherigen Technologien erlangt hat.

\begin{figure}
	\centering
	\caption{Verteilung von Publikationen zu \glqq Smart Robots\grqq}
	\includegraphics[width=\linewidth,clip,trim=0cm 0.5cm 1cm 0.8cm,height=\textheight/3]{pdf/SR.pdf}
	\caption*{Quelle: Eigene Darstellung}
	\label{fig:sr_pub}
\end{figure}

TODO: DIstributed Ledger: Die \glqq Blockchain-Technologie\grqq~

\begin{figure}
	\centering
	\caption{Verteilung von Publikationen zu \glqq Blockchain\grqq}
	\includegraphics[width=\linewidth,clip,trim=0cm 0.5cm 1cm 0.8cm,height=\textheight/3]{pdf/B.pdf}
	\caption*{Quelle: Eigene Darstellung}
	\label{fig:b_pub}
\end{figure}

\glqq Connected Home\grqq~gehört ebenfalls zu den Technologien, die in der Wissenschaft eine gewisse Beachtung finden. In Abbildung \ref{fig:ch_pub} ist zu erkennen, dass schon zu Beginn des Betrachtungszeitraumes bei \ac{IEEE} und \ac{ACM} mit etwa 150 Konferenzbeiträgen ein relativ hohes Interesse an der Technologie vorhanden ist. Die Tendenz an Publikationen ist, ausgenommen von Konferenzbeiträgen im \ac{ACM}, bis 2017 steigend. Sie erreicht ihren Höhepunkt im Jahre 2017 und liegt für Konferenzbeiträge im \ac{IEEE} und Fachartikel im \ac{WoS} bei ca. 400 Veröffentlichungen. Dazu war es allerdings notwendig, den Technologiebegriff bei der Suche um die gängigen Synonyme zu erweitern.

\begin{figure}
	\centering
	\caption{Verteilung von Publikationen zu \glqq Connected Home\grqq}
	\includegraphics[width=\linewidth,clip,trim=0cm 0.5cm 1cm 0.8cm,height=\textheight/3]{pdf/CH.pdf}
	\caption*{Quelle: Eigene Darstellung}
	\label{fig:ch_pub}
\end{figure}

Die Technologie \glqq \acl{ML}\grqq~hat mit Abstand die höchsten Publikationszahlen unter den untersuchten Technologien. Die Trends in den einzelnen Ausprägungen von Publikationen sind jedoch unterschiedlich. Abbildung \ref{fig:ml_pub} zeigt, dass Konferenzbeiträge im \ac{IEEE} sowie Fachartikel im \ac{WoS} im Betrachtungszeitraum ein starkes positives Wachstum erfahren und von 2014 an nahezu parallel und in Tausenderschritten die Anzahl von über \numprint{5000} Publikationen erreichen. Beide Veröffentlichungstypen im \ac{ACM} pendeln sich ab 2016 bei ca. \numprint{1000} Veröffentlichungen ein. Fachartikel im \ac{IEEE} hingegen erreichen eine Anzahl von ungefähr 800 im Jahre 2017 nach zuvor mäßigem Wachstum. Allein die Konferenzbeiträge im \ac{WoS} fallen -- absolut betrachtet -- auffallend gering aus. Doch auch hier handelt es sich um die höchsten Werte über alle Konferenzbeiträge im \ac{WoS}. Dabei wurde die durch Ausschluss der Nachfolgetechnologie \glqq Deep Learning\grqq~eingeschränkt.

\begin{figure}
	\centering
	\caption{Verteilung von Publikationen zu \glqq Machine Learning\grqq}
	\includegraphics[width=\linewidth,clip,trim=0cm 0.5cm 1cm 0.8cm,height=\textheight/3]{pdf/ML.pdf}
	\caption*{Quelle: Eigene Darstellung}
	\label{fig:ml_pub}
\end{figure}

Abbildung \ref{fig:sds_pub} stellt den Verlauf der Technologie \glqq Software-Defined Security\grqq~dar. Die Publikationen dazu sind marginal. Einzig nennenswert ist die steigende Tendenz für Konferenzbeiträge im \ac{IEEE}, die allerdings mit zehn Publikationen in 2017 schon den Höchstwert ausmacht.

\begin{figure}
	\centering
	\caption{Verteilung von Publikationen zu \glqq Software-Defined Security\grqq}
	\includegraphics[width=\linewidth,clip,trim=0cm 0.5cm 1cm 0.8cm,height=\textheight/3]{pdf/SDS.pdf}
	\caption*{Quelle: Eigene Darstellung}
	\label{fig:sds_pub}
\end{figure}

Der Trendverlauf der Technologie \glqq Autonomous Vehicles\grqq~ist Abbildung \ref{fig:av_pub} zu entnehmen. Erneut heben sich die Kurven der Konferenzberichte im \ac{IEEE} und Fachartikel im \ac{WoS} den anderen gegenüber ab. Erstere steigen ab 2015 stärker an als in der Jahren davor und erreichen in 2017 auf 675 Veröffentlichungen. Die Fachartikel im \ac{WoS} kommen auf 361 im selben Jahr. Konferenzberichte im \ac{WoS} sind auch hier kaum vorhanden. Die anderen Veröffentlichungen bewegen sich mit leicht steigender Tendenz um die Anzahl von 100 herum.

\begin{figure}
	\centering
	\caption{Verteilung von Publikationen zu \glqq Autonomous Vehicles\grqq}
	\includegraphics[width=\linewidth,clip,trim=0cm 0.5cm 1cm 0.8cm,height=\textheight/3]{pdf/AV.pdf}
	\caption*{Quelle: Eigene Darstellung}
	\label{fig:av_pub}
\end{figure}

Schließlich zeigt Abbildung \ref{fig:ne_pub} die Trendentwicklung der Technologie \glqq Nanotube Electronics\grqq. Erneut ist zu sehen, dass Konferenzberichte im \ac{IEEE} und Fachartikel im \ac{WoS} die übrigen Publikationen dominieren. Sie bewegen sich im Betrachtungszeitraum um 100 Publikationen herum, allerdings mit einer leicht fallenden Tendenz. Im Jahre 2016 ist ein leichter Ausschlag nach oben zu erkennen. Konferenzberichte im \ac{WoS} sind wieder kaum vorhanden. Der Rest befindet sich relativ konstant bei um die 20 Veröffentlichungen.

\begin{figure}
	\centering
	\caption{Verteilung von Publikationen zu \glqq Nanotube Electronics\grqq}
	\includegraphics[width=\linewidth,clip,trim=0cm 0.5cm 1cm 0.8cm,height=\textheight/3]{pdf/NE.pdf}
	\caption*{Quelle: Eigene Darstellung}
	\label{fig:ne_pub}
\end{figure}

Da die Technologien \glqq Cognitive Expert Advisors\grqq~und \glqq Software-Defined Anything\grqq~ -- auch über den Betrachtungszeitraum hinaus -- kaum nennenswert in wissenschaftlichen Publikationen vorkommen, werden sie nicht visualisiert.

\subsection{Klassifizierung der Technologien}
Durch die deskriptive Auswertung der Daten sind Muster zu erkennen, die eine Klassifizierung der Technologien ermöglichen. Sie dient der besseren Vergleichbarkeit des Trendverhältnisses zwischen Wirtschaft und Wissenschaft.

In Tabelle \ref{tab:class_trend_ev} sind die Technologien den Trendstufen zugeordnet. Die zeitliche Verteilung dazu ist in den Zellen durch die Angabe des entsprechenden Zeitraumes bzw.~-punktes abgebildet.

\begin{table}
	\caption{Zeitliche Zuordnung von Technologien zu Trendstufen}
	\fontfamily{pcr}\selectfont
	\footnotesize
	\centering
	\label{tab:class_trend_ev}
	\begin{tabularx}{\linewidth}{p{2.2cm}XXXXX}
		Technologie & $>$ 2000 & $>$ 500 & $>$ 100 & $>$ 20 & $<$ 20 \\
		\hline
		\acs{GCD} & - & - & 2014 & 2011-2013 2015-2017 & 2010 \\
		\acs{MDC} & - & - & - & - & 2011-2017 \\
		\acs{SR} & - & 2010-2017 & - & - & - \\
		\acs{B} & - & & 2016 & - & 2013-2015 \\
		\acs{CH} & - & 2012-2017 & 2010-2011 & - & - \\
		\acs{CEA} & - & - & - & - & 2017 \\
		\acs{ML} & 2010-2017 & - & - & - & - \\
		\acs{SDS} & - & - & - & - & 2014-2017 \\
		\acs{AV} & - & 2015-2017 & 2010-2014 & - & - \\
		\acs{NE} & - & - & 2010-2017 & - & - \\
		\acs{SDx} & - & - & - & - & 2017 \\
		\hline
	\end{tabularx}
	\caption*{Quelle: Eigene Darstellung}
\end{table}

Tabelle \ref{tab:class_tech} zeigt die Zuordnung der Technologien zu ihren Rollen. 

\begin{table}
	\caption{Fachliche Klassifizierung der Technologien}
	\fontfamily{pcr}\selectfont
	\footnotesize
	\centering
	\label{tab:class_tech}
	\begin{tabularx}{\linewidth}{XXXX}
		Technologie & Produkt & Infrastruktur & Komponente \\
		\hline
		\acs{GCD} & $\checkmark$ & & \\
		\hline
		\acs{MDC} & & $\checkmark$ & \\
		\hline
		\acs{SR} & $\checkmark$ & & \\
		\hline
		\acs{B} & & $\checkmark$ & \\
		\hline
		\acs{CH} & $\checkmark$ & & \\
		\hline
		\acs{CEA} & & & $\checkmark$ \\
		\hline
		\acs{ML} & & & $\checkmark$ \\
		\hline
		\acs{SDS} & & $\checkmark$ & \\
		\hline
		\acs{AV} & $\checkmark$ & & \\
		\hline
		\acs{NE} & & & $\checkmark$ \\
		\hline
		\acs{SDx} & & $\checkmark$ & \\
		\hline
	\end{tabularx}
	\caption*{Quelle: Eigene Darstellung}
\end{table}