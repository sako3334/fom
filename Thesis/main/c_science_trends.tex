\section{Technologietrends in der akademischen Forschung}
\subsection{Wissenschaftliche Fachzeitschriften}
Wissenschaftliche Publikationen sind für die Kommunikation und Verbreitung wissenschaftlicher Forschungsergebnisse von essenzieller Bedeutung und leisten einen wesentlichen Beitrag zum Erfolg des Forschenden.\footnote{\citeNP<Vgl.>[S.~417.]{Turbek2016}}

Sie werden hauptsächlich in regelmäßig erscheinenden wissenschaftlichen Fachzeitschriften veröffentlicht, in der Regel in englischer Sprache über einen Drittverlag und nicht der Universität des Autors.\footnote{\citeNP<Vgl.>[o. S.]{Bjork2009}} Zuvor wird ein \glqq Peer-Review\grqq~durch unabhängige Gutachter des gleichen Fachgebietes durchgeführt, bei dem die Qualität und Gültigkeit der Arbeit überprüft werden.\footnote{\citeNP<Vgl.>[S.~707.]{Thurner2011}}

Auch hinsichtlich Form, Umfang, Zitierweise und Schreibstil folgen sie meist bewährten Konventionen.\footnote{\citeNP<Vgl.>[o. S.]{Bjork2009}}

Einer Hochrechnung von \shortciteauthor{Jinha2010} zufolge wurden zwischen der ersten Veröffentlichung eines wissenschaftlichen Artikels im Jahre 1665 in Frankreich und dem Jahr der Studie 2009 insgesamt rund 50 Millionen Forschungsergebnisse nach einem \glqq Peer-Review\grqq~in wissenschaftlichen Artikeln publiziert. Die Verteilung der Publikationen über den analysierten Zeitraum zeigt dabei ein exponentielles Wachstum.\footnote{\citeNP<Vgl.>[S.~261f.]{Jinha2010}}

Im Zeitalter des Internets sind wissenschaftliche Publikationen meist in digitaler Form verfügbar, obgleich der überwiegende Anteil für die Allgemeinheit kostenpflichtig ist.\footnote{\citeNP<Vgl.>[o. S.]{Bjork2009}} Die Entwicklung von wissenschaftlichen Datenbanken schon in der frühen Phase des Internets zeigt, wie wichtig die Speicherung und Verbreitung von Forschungsergebnissen für die Wissenschaft ist.\footnote{\citeNP<Vgl.>[S.338.]{Falagas2007}}

Sie bieten erweiterte Funktionen zur Suche und Bibliometrie von Artikeln, und werden daher bei der Literaturrecherche ausgiebig eingesetzt, um bereits existierende Forschungsergebnisse auszuwerten. \footnote{\citeNP<Vgl.>[S.1320.]{Archambault2009}}

\subsection{Bedeutung von Publikationen für die Forschung}
Die Verbreitung von Wissen, das mittels akademischer Forschung gewonnenen wurde, erfolgt hauptsächlich über zwei Kanäle: Lehre und Publikationen in wissenschaftlichen Zeitschriften. Letzterer gewinnt dabei immer mehr an Bedeutung, da er vor allem für Forscher, die am Anfang ihrer Laufbahn stehen, bessere Möglichkeiten für ihre Karriere bietet.\footnote{\citeNP<Vgl.>[S.322f.]{DeRond2005}}

Der Aphorismus \glqq Publish or Perish\grqq\footnote{Englisch sinngemäß für \glqq Publizieren oder Untergehen\grqq.} verdeutlicht den Druck, der auf Wissenschaftlern lastet, relevante Forschungsergebnisse in renommierten Fachzeitschriften zu veröffentlichen, um die nötige Aufmerksamkeit auf die eigenen Fähigkeiten zu lenken und damit die Karriereoptionen zu verbessern.\footnote{\citeNP<Vgl.>[S.391.]{Clapham2005}}

Eine kontinuierliche Veröffentlichung von Forschungsergebnissen kann auch für Unternehmen nützlich sein, da Innovationen, die von Wettbewerbern parallel entwickelt und patentiert werden, rechtlich weiterhin nutzbar bleiben.\footnote{\citeNP<Vgl.>[S.927.]{Parchomovsky2000}}

\subsection{Trends in der akademischen Forschung}
Forschungstrends in der wissenschaftlichen Fachliteratur können sowohl interne als auch externen Ursachen haben. Typische interne Ursachen sind neue Entdeckungen und wissenschaftliche Durchbrüche durch andere Forscher, die erhöhte Forschungspotentiale auslösen können. Externe Ursachen sind natürliche bzw. gesellschaftliche Ereignisse, die unabhängig von der Forschung eintreten und Wissenschaftler dazu anregen können, ein Thema aus neuen Perspektiven zu untersuchen.\footnote{\citeNP<Vgl.>[S.359.]{Chen2006}}

Dabei sind kontinuierliche Verbesserungen an bestehenden Technologien von disruptiven Technologien zu unterscheiden.\footnote{\citeNP<Vgl.>[S.114.]{Dotsika2017}}

\shortciteauthor{Price1965} konnte ein zeitliches Muster von Forschungsschwerpunkten zu einem bestimmten Thema beobachten, die unmittelbar aufeinander folgten, folglich die meist zitierten Artikel gleichzeitig auch die aktuellsten waren.\footnote{\citeNP<Vgl.>[S.512f.]{Price1965}} Dieser Effekt gibt eine gute Erklärung für das bekannte Phänomen, dass Artikel wenige Jahre nach ihrer Veröffentlichung als veraltet gelten.\footnote{\citeNP<Vgl.>[S.360.]{Chen2006}}



\subsection{Schnittpunkte mit der praktizierenden Wirtschaft}
Today, public resources for academic research are
not increasing significantly, so industrial collaboration is seen by many academics as a strategy to gain increased financial support for their research
Calvert, Jane
Patel, Parimal