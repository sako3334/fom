\section{Technologietrends in der akademischen Forschung}
\subsection{Wissenschaftliche Fachzeitschriften}
Wissenschaftliche Publikationen sind für die Kommunikation und Verbreitung wissenschaftlicher Forschungsergebnisse von essenzieller Bedeutung und leisten einen wesentlichen Beitrag zum Erfolg des Forschenden.\footnote{\citeNP<Vgl.>[S.~417.]{Turbek2016}}

Sie werden hauptsächlich in regelmäßig erscheinenden wissenschaftlichen Fachzeitschriften veröffentlicht, in der Regel in englischer Sprache über einen Drittverlag und nicht der Universität des Autors.\footnote{\citeNP<Vgl.>[o. S.]{Bjork2009}} Zuvor wird ein \glqq Peer-Review\grqq~durch unabhängige Gutachter des gleichen Fachgebietes durchgeführt, bei dem die Qualität und Gültigkeit der Arbeit überprüft werden.\footnote{\citeNP<Vgl.>[S.~707.]{Thurner2011}}

Auch hinsichtlich Form, Umfang, Zitierweise und Schreibstil folgen sie meist bewährten Konventionen.\footnote{\citeNP<Vgl.>[o. S.]{Bjork2009}}

Einer Hochrechnung von \shortciteauthor{Jinha2010} zufolge wurden zwischen der ersten Veröffentlichung eines wissenschaftlichen Artikels im Jahre 1665 in Frankreich und dem Jahr der Studie 2009 insgesamt rund 50 Millionen Forschungsergebnisse nach einem \glqq Peer-Review\grqq~in wissenschaftlichen Artikeln publiziert. Die Verteilung der Publikationen über den analysierten Zeitraum zeigt dabei ein exponentielles Wachstum.\footnote{\citeNP<Vgl.>[S.~261f.]{Jinha2010}}

Im Zeitalter des Internets sind wissenschaftliche Publikationen meist in digitaler Form verfügbar, obgleich der überwiegende Anteil für die Allgemeinheit kostenpflichtig ist.\footnote{\citeNP<Vgl.>[o. S.]{Bjork2009}} Die Entwicklung von wissenschaftlichen Datenbanken schon in der frühen Phase des Internets zeigt, wie wichtig die Speicherung und Verbreitung von Forschungsergebnissen für die Wissenschaft ist.\footnote{\citeNP<Vgl.>[S.338.]{Falagas2007}}

Sie bieten erweiterte Funktionen zur Suche und Bibliometrie von Artikeln, und werden daher bei der Literaturrecherche ausgiebig eingesetzt, um bereits existierende Forschungsergebnisse auszuwerten.\footnote{\citeNP<Vgl.>[S.1320.]{Archambault2009}}
\subsection{Zweck von wissenschaftlichen Artikeln}


\subsection{Trends in der akademischen Forschung}
\subsection{Schnittpunkte mit der praktizierenden Wirtschaft}
Today, public resources for academic research are
not increasing significantly, so industrial collaboration is seen by many academics as a strategy to gain increased financial support for their research
Calvert, Jane
Patel, Parimal