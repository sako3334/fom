\section{Technologietrends in der akademischen Forschung}
\subsection{Wissenschaftliche Publikationen}\label{sec:acad_pub}
Wissenschaftliche Publikationen sind für die Kommunikation und Verbreitung wissenschaftlicher Forschungsergebnisse von essenzieller Bedeutung und leisten einen wesentlichen Beitrag zum Erfolg des Forschenden.\footnote{\citeNP<Vgl.>[S.~417.]{Turbek2016}}

Sie werden hauptsächlich in regelmäßig erscheinenden wissenschaftlichen Fachzeitschriften \footnote{engl. Journals} veröffentlicht, in der Regel in englischer Sprache über einen Drittverlag und nicht der Universität des Autors.\footnote{\citeNP<Vgl.>[o. S.]{Bjork2009}} Zuvor wird ein \glqq Peer-Review\grqq~durch unabhängige Gutachter des gleichen Fachgebietes durchgeführt, bei dem die Qualität und Gültigkeit der Arbeit überprüft werden.\footnote{\citeNP<Vgl.>[S.~707.]{Thurner2011}} Auch hinsichtlich Form, Umfang, Zitierweise und Schreibstil folgen sie meist bewährten Konventionen.\footnote{\citeNP<Vgl.>[o. S.]{Bjork2009}}

Daneben werden neue Erkenntnisse aus wissenschaftlichen Studien, insbesondere der Wissenschaften rund um IT-Technologien, in wissenschaftlichen Konferenzbeiträgen\footnote{engl. Conference proceedings} veröffentlicht.\footnote{\citeNP<Vgl.>[S.~810.]{Bar-Ilan2010}} Dadurch werden Innovationen früher mit der akademischen Gemeinschaft kommuniziert und häufig, abhängig von der Resonanz, in erweiterter und ausgereifterer Form in Fachzeitschriften publiziert.\footnote{\citeNP<Vgl.>[S.~1319.]{Zhang2011}}

Einer Hochrechnung von \shortciteauthor{Jinha2010} zufolge wurden zwischen der ersten Veröffentlichung eines wissenschaftlichen Artikels im Jahre 1665 in Frankreich und dem Jahr der Studie 2009 insgesamt rund 50 Millionen Forschungsergebnisse nach einem \glqq Peer-Review\grqq~in wissenschaftlichen Artikeln publiziert. Die Verteilung der Publikationen über den analysierten Zeitraum zeigt dabei ein exponentielles Wachstum.\footnote{\citeNP<Vgl.>[S.~261f.]{Jinha2010}}

Im Zeitalter des Internets sind wissenschaftliche Publikationen meist in digitaler Form verfügbar, obgleich der überwiegende Anteil für die Allgemeinheit kostenpflichtig ist.\footnote{\citeNP<Vgl.>[o. S.]{Bjork2009}} Die Entwicklung von wissenschaftlichen Datenbanken schon in der frühen Phase des Internets zeigt, wie wichtig die Speicherung und Verbreitung von Forschungsergebnissen für die Wissenschaft sind.\footnote{\citeNP<Vgl.>[S.~338.]{Falagas2007}}

Sie bieten erweiterte Funktionen zur Suche und Bibliometrie von Artikeln, und werden daher bei der Literaturrecherche ausgiebig eingesetzt, um bereits existierende Forschungsergebnisse auszuwerten.\footnote{\citeNP<Vgl.>[S.~1320.]{Archambault2009}}

\subsection{Bedeutung von Publikationen für die Forschung}
Die Verbreitung von Wissen, das mittels akademischer Forschung gewonnenen wurde, erfolgt hauptsächlich über zwei Kanäle: Lehre und Publikationen in wissenschaftlichen Zeitschriften. Letzterer gewinnt dabei immer mehr an Bedeutung, da er vor allem für Forscher, die am Anfang ihrer Laufbahn stehen, bessere Möglichkeiten für ihre Karriere bietet.\footnote{\citeNP<Vgl.>[S.~322f.]{DeRond2005}}

Der Aphorismus \glqq Publish or Perish\grqq\footnote{Englisch sinngemäß für \glqq Publizieren oder Untergehen\grqq.} verdeutlicht den Druck, der auf Wissenschaftlern lastet, relevante Forschungsergebnisse in renommierten Fachzeitschriften zu veröffentlichen, um die nötige Aufmerksamkeit auf die eigenen Fähigkeiten zu lenken und damit die Karriereoptionen zu verbessern.\footnote{\citeNP<Vgl.>[S.~391.]{Clapham2005}}

Eine kontinuierliche Veröffentlichung von Forschungsergebnissen kann auch für Unternehmen nützlich sein, da Innovationen, die von Wettbewerbern parallel entwickelt und patentiert werden, rechtlich weiterhin nutzbar bleiben.\footnote{\citeNP<Vgl.>[S.~927.]{Parchomovsky2000}}

\subsection{Trends in der akademischen Forschung}
Forschungstrends in der wissenschaftlichen Fachliteratur können sowohl interne als auch externe Ursachen haben. Typische interne Ursachen sind neue Entdeckungen und wissenschaftliche Durchbrüche durch andere Forscher, die erhöhte Forschungspotentiale auslösen können. Externe Ursachen sind natürliche bzw. gesellschaftliche Ereignisse, die unabhängig von der Forschung eintreten und Wissenschaftler dazu anregen können, ein Thema aus neuen Perspektiven zu untersuchen.\footnote{\citeNP<Vgl.>[S.~359.]{Chen2006}}

Bei den internen Ursachen sind inkrementelle Verbesserungen an bestehenden Technologien von revolutionär neuen Technologien zu unterscheiden.\footnote{\citeNP<Vgl.>[S.~114.]{Dotsika2017}} Letztere haben weitreichendere Auswirkungen auf das sozioökonomische System und werden in der Literatur oft als \glqq Emerging Technologies\grqq~bzw. \glqq Disruptive Technologies\grqq~bezeichnet.\footnote{\citeNP<Vgl.>[S.~285.]{Li2018}} Der Unterschied besteht in der anwendungsorientierten Reife der neuen Technologie. Bei einer \glqq Emerging Technology\grqq~muss der praktische Nutzen einer theoretischen Erkenntnis zunächst festgestellt werden. Die sog. \glqq Disruptive Technology\grqq~hingegen hat diese Phase bereits durch eine signifikante finanzielle oder technologische Verbesserung positiv gemeistert.\footnote{\citeNP<Vgl.>[S.~294.]{Li2018}}

Diese Art von Technologien haben einen potentiell stärkeren Effekt auf Forschungstrends in der Wissenschaft als solche, die auf Basis vorhandener Technologien mit marginalen Verbesserungen ausgestattet wurden.

\shortciteauthor{Price1965} zufolge kann ein Muster für Forschungsschwerpunkte zu einem bestimmten Thema beobachtet werden. Neue wissenschaftliche Artikel beziehen sich demnach eher auf die aktuellsten Publikationen als auf deren Vorgänger. Im Umkehrschluss bedeutet das, dass die meist zitierten Artikel gleichzeitig auch die aktuellsten sind.\footnote{\citeNP<Vgl.>[S.~512f.]{Price1965}} Dieser Effekt gibt eine gute Erklärung für das bekannte Phänomen, dass Artikel wenige Jahre nach ihrer Veröffentlichung als veraltet gelten.\footnote{\citeNP<Vgl.>[S.~360.]{Chen2006}}

Zur Ermittlung des Verlaufes einer Technologie in der Wissenschaft kommt häufig die Literaturanalyse von wissenschaftlichen Publikationen zum Einsatz.\footnote{\citeNP<Vgl.>[S.~149.]{Osareh1996}} Sie wird überwiegend in Form von Patentanalysen und Bibliometrie durchgeführt.\footnote{\citeNP<Vgl.>[S.~983f.]{Daim2006}}\footnote{\citeNP<Vgl.>[S.~387f.]{Wu2011}}\footnote{\citeNP<Vgl.>[S.~269.]{Chao2007}}

Bei der Patentanalyse werden die Patente klassifiziert und Technologien zugeordnet. Die Abhängigkeit der Anzahl zum Zeitpunkt der Patentanmeldung macht eine Trendanalyse möglich.\footnote{\citeNP<Vgl.>[S.~983.]{Daim2006}}

\label{sec:biblio}
Die bibliometrische Analyse ist vom Prinzip ähnlich. Für die Klassifizierung werden allerdings die Metadaten, wie Titel, Stichwörter und Abstract genutzt, da sie für die Bestimmung des Themas eines Artikels ausschlaggebend sind.\footnote{\citeNP<Vgl.>[S.~269.]{Chao2007}} Die Suche nach einem Begriff im Volltext ist deshalb problematisch, da ihre bloße Erwähnung beispielsweise in einem Exkurs zum Treffer führen würde. In den Metadaten kommen laut Definition keine Abschweifungen vom Thema vor, weshalb sie sich besser eignen. Für die Auswertung wird ebenfalls die Anzahl an Publikationen verwendet und kann durch die der Referenzen von Zitaten ergänzt werden.\footnote{\citeNP<Vgl.>[S.~2215.]{Bornmann2015}} Diese sog. Zitationsanalyse wird jedoch aufgrund einer Vielzahl von Schwächen oft kritisiert.\footnote{\citeNP<Vgl.>[S.~2105.]{Meho2007}}

\subsection{Schnittstellen mit der praktizierenden Wirtschaft}
Nach \shortciteauthor{Merton1957} ist das Hauptmotiv für die klassische wissenschaftliche Forschung die Anerkennung durch die akademische Gemeinschaft, eine neue Entdeckung vor allen anderen Forschern veröffentlicht zu haben.\footnote{\citeNP<Vgl.>[S.~659.]{Merton1957}}

Im Zuge der Globalisierung haben damit einhergehende sozioökonomische Herausforderungen Initiativen hervorgerufen, die wissenschaftliche Wissensbasis für Innovation und wirtschaftliche Wettbewerbsfähigkeit zu nutzen, indem die Zusammenarbeit zwischen Universität und Industrie verstärkt wird.\footnote{\citeNP<Vgl.>[S.~1355.]{Lam2011}}

Folglich fand eine Kommerzialisierung des akademischen Wissens, bspw. in Form von Patenten und Lizenzen für Erfindungen, statt.\footnote{\citeNP<Vgl.>[S.~423f.]{Perkmann2013}}

Auch die Stagnation von öffentlichen Mitteln für die akademische Forschung hat die industrielle Zusammenarbeit von vielen Wissenschaftlern vorangetrieben, um mehr finanzielle Unterstützung für ihre Forschung zu erhalten.\footnote{\citeNP<Vgl.>[S.~86.]{Calvert2003}}