\section{Technologietrends in der praktizierenden Wirtschaft}


\subsection{Sicht auf Technologien}
Unternehmen bewegen sich in einem Spannungsfeld zwischen Investitionen in neue Technologien, um wettbewerbsfähig zu bleiben und optimaler Auslastung ihrer Ressourcenkapazitäten für bestehende Projekte. Denn die Erfolg versprechendste Technologie ist wirkungslos, wenn sie nicht zu einer geeigneten Zeit und mit richtigen Methoden eingesetzt wird.\footnote{\citeNP<Vgl.>[S.~518]{Dickinson2001}.}

Unter dem Begriff F\&E (Forschung und Entwicklung) werden die dafür notwendigen Maßnahmen bezeichnet, die strukturiert in einem Unternehmen gesteuert werden und die Gewinnung neuen Wissens im Bezug auf Technologien als Ziel verfolgen. Dazu werden Ressourcen des Unternehmens, wie etwa Mitarbeiter, monetäre Mittel etc. bereitgestellt, die zunächst einmal keinen unmittelbaren Ertrag herbeiführen, jedoch für die strategische Planung einen wichtigen Beitrag liefern. Ebenso werden Erkenntnisse über die unternehmenseigenen Stärken und Schwächen sowie externe Einflüsse des Marktes in Form von Chancen und Risiken gewonnen.\footnote{\citeNP<Vgl.>[S.~383f]{Jolly2003}.}

Nach Porter gehören F\&E deshalb zu den unterstützenden und nicht zu den primären Aktivitäten der Wertschöpfungskette. Dennoch hält er Technologien innerhalb eines Unternehmens für allgegenwärtig, also mehr oder weniger aller Aktivitäten inhärent. Sie erstrecken sich darüber hinaus auf Lieferanten, Kunden und die Vertriebspolitik, weshalb er hierfür den breiter gefassten Begriff Technologieentwicklung vorzieht.\footnote{\citeNP<Vgl.>[S.~36--42]{Porter1985}.}

Als Unterstützung für den Entscheidungsprozess, welche Technologien potentiell wertvoll im Sinne der Wettbewerbsfähigkeit sind und mit welchen Mitteln die Technologieentwicklung stattfinden sollte, wurden bereits in den 1980er-Jahren Konzepte für das Portfoliomanagement von Technologien entwickelt. Sie sollen die begrenzten Ressourcen eines Unternehmens ausgerichtet an Risiken, Erträgen und der Unternehmensstrategie effizient verteilen.\footnote{\citeNP<Vgl.>[S.~383f]{Jolly2003}.} 

Die wichtigsten Kriterien für die Auswahl der zu untersuchenden bzw. einzusetzenden Technologien ergeben sich aus der Auswertung beeinflussbarer und unbeeinflussbarer Faktoren. Bei den ersteren handelt es sich um firmenspezifische Merkmale sowie den Umgang mit der Wettbewerbssituation.\footnote{\citeNP<Vgl.>[S.~188]{Sinha2005}.} Die Fachkompetenz der Mitarbeiter kann beispielsweise durch Fortbildungen beeinflusst werden. Ebenso obliegt die Entscheidung über strategische Maßnahmen zur Konkurrenzabwehr in der eigenen Hand. Bei den unbeeinflussbaren Faktoren spielt vor allem die Marktsituation eine Rolle. Zum Beispiel ist in schnell wachsenden Marktsegmenten mit einer Vielzahl an Konkurrenten der Zugzwang Alleinstellungsmerkmale zu schaffen durch den Markt vorgegeben. Auch politische Entscheidungen, die direkt oder indirekt auf das eigene Handeln einwirken, liegt in der Regel nicht im Entscheidungsfreiraum eines Unternehmens.\footnote{\citeNP<Vgl.>[S.~188f]{Sinha2005}.}

Aus der Auswertung dieser Kriterien resultiert eine Tendenz, die Wahrscheinlichkeitsangaben für den bestmöglichen Zeitpunkt des möglichen Einsatzes einer Technologie zulässt.\footnote{\citeNP<Vgl.>[S.~519]{Dickinson2001}.} Basierend darauf können im Hinblick auf potentielle Risiken und Gewinne Investitionen getätigt werden.\footnote{\citeNP<Vgl.>[S.~804]{Dolci2014}.}

\subsection{Einflussfaktoren für Technologietrends}

Technology driven markets


Time to market

Standards

\subsection{Informationsquellen von Unternehmern}
Gartner, Forrester, Aberdeen, and IDC

\subsection{Gartner’s Hype Cycle for Emerging Technologies}

\subsection{Schnittpunkte zur akademischen Forschung}
science-to-business