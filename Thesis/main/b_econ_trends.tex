\section{Technologietrends in der praktizierenden Wirtschaft}


\subsection{Sicht auf Technologien}
Unternehmen bewegen sich in einem Spannungsfeld zwischen Investitionen in neue Technologien, um wettbewerbsfähig zu bleiben und optimaler Auslastung ihrer Ressourcenkapazitäten für bestehende Projekte. Denn die Erfolg versprechendste Technologie ist wirkungslos, wenn sie nicht zu einer geeigneten Zeit und mit richtigen Methoden Anwendung findet. Deshalb wurden bereits in den 1980er-Jahren Konzepte zum Portfoliomanagement entwickelt, welche die begrenzten Ressourcen eines Unternehmens ausgerichtet an Risiken, Erträgen und der Unternehmensstrategie optimal einsetzen sollen.\footnote{\citeNP<Vgl.>[S.~518]{Dickinson2001}.} 



\footnote{\citeNP<Vgl.>[S.~383f]{Jolly2003}.}

\subsection{Einflussfaktoren für Technologietrends}
\subsection{Informationsquellen von Unternehmern}
\subsection{Gartner’s Hype Cycle for Emerging Technologies}
\subsection{Schnittpunkte zur akademischen Forschung}