\section{Technologietrends in der praktizierenden Wirtschaft}


\subsection{Sicht auf Technologien}
Unternehmen bewegen sich in einem Spannungsfeld zwischen Investitionen in neue Technologien, um wettbewerbsfähig zu bleiben und optimaler Auslastung ihrer Ressourcenkapazitäten für bestehende Projekte. Denn die Erfolg versprechendste Technologie ist wirkungslos, wenn sie nicht zu einer geeigneten Zeit und mit richtigen Methoden eingesetzt wird.\footnote{\citeNP<Vgl.>[S.~518]{Dickinson2001}.}

Mit dem Begriff F\&E (Forschung und Entwicklung) werden im Unternehmen die Aktivitäten bezeichnet, mit denen die Gewinnung neuen Wissens im Bezug auf Technologien ermöglicht wird. Dabei werden Ressourcen des Unternehmens, wie etwa Mitarbeiter, monetäre Mittel etc. bereitgestellt, die zunächst einmal keinen unmittelbaren Ertrag herbeiführen, jedoch für die strategische Planung einen wichtigen Beitrag liefern.\footnote{\citeNP<Vgl.>[S.~383f]{Jolly2003}.} 

Nach Porter gehören F\&E deshalb zu den unterstützenden und nicht zu den primären Aktivitäten der Wertschöpfungskette. Dennoch ist die Technologie innerhalb eines Unternehmen allgegenwärtig und erstreckt sich darüber hinaus auf Lieferanten, Kunden und die Vertriebspolitik. Deshalb bevorzugt er den breiter gefassten Begriff Technologieentwicklung.\footnote{\citeNP<Vgl.>[S.~36--42]{Porter1985}.}

Dabei werden ebenso Erkenntnisse über die unternehmenseigenen Stärken und Schwächen sowie externe Einflüsse des Marktes in Form von Chancen und Risiken gewonnen.


%Portfoliomgmt\footnote{\citeNP<Vgl.>[S.~215]{Lawson1985}.}
Deshalb wurden bereits in den 1980er-Jahren Konzepte zum Portfoliomanagement entwickelt, welche die begrenzten Ressourcen eines Unternehmens ausgerichtet an Risiken, Erträgen und der Unternehmensstrategie optimal verteilen sollen.

\subsection{Einflussfaktoren für Technologietrends}
\subsection{Informationsquellen von Unternehmern}
\subsection{Gartner’s Hype Cycle for Emerging Technologies}
\subsection{Schnittpunkte zur akademischen Forschung}