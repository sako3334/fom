\section{Technologietrends in der praktizierenden Wirtschaft}

\subsection{Sicht auf Technologien}\label{sec:sight}
Der Begriff Technologie lässt sich als eine spezielle Form des Wissens abstrahieren und entsteht dort, wo dieses Fachwissen Anwendung findet. Deshalb wird sie eher als ihre physikalische Ausprägung wie etwa in Form von Produkten und Systemen wahrgenommen.\footnote{\citeNP<Vgl.>[S.~6f.]{Phaal2004}} Da somit die Aneignung des theoretischen Wissens mit ihrer Implementierung in die Praxis zusammentrifft, ist die Entwicklung neuer Technologien mit (Opportunitäts-)Kosten verbunden.

Unternehmen bewegen sich deshalb in einem Spannungsfeld zwischen Investitionen in neue Technologien, um wettbewerbsfähig zu bleiben und optimaler Auslastung ihrer Ressourcenkapazitäten für den laufenden Betrieb. Denn die Erfolg versprechendste Technologie ist wirkungslos, wenn sie nicht zu einer geeigneten Zeit und mit richtigen Methoden eingesetzt wird.\footnote{\citeNP<Vgl.>[S.~518.]{Dickinson2001}}

Unter dem Begriff F\&E (Forschung und Entwicklung) werden die dafür notwendigen Maßnahmen bezeichnet, die strukturiert in einem Unternehmen gesteuert werden und die Gewinnung neuen Wissens im Bezug auf Technologien als Ziel verfolgen. Dazu werden Ressourcen des Unternehmens, wie etwa Mitarbeiter, monetäre Mittel etc. bereitgestellt, die zunächst einmal keinen unmittelbaren Ertrag herbeiführen, jedoch für die strategische Planung einen wichtigen Beitrag liefern. Ebenso werden Erkenntnisse über die unternehmenseigenen Stärken und Schwächen sowie externe Einflüsse des Marktes in Form von Chancen und Risiken gewonnen.\footnote{\citeNP<Vgl.>[S.~383f.]{Jolly2003}} Es handelt sich somit um eine Management-Aufgabe, welche unter ständiger Berücksichtigung der Unternehmensziele die Gewinnung, Nutzung und den Schutz von Wissen anstrebt.

Nach Porter gehören F\&E deshalb zu den unterstützenden und nicht zu den primären Aktivitäten der Wertschöpfungskette. Dennoch hält er Technologien innerhalb eines Unternehmens für allgegenwärtig, also mehr oder weniger aller Aktivitäten inhärent. Sie erstrecken sich darüber hinaus auf Lieferanten, Kunden und die Vertriebspolitik, weshalb er hierfür den breiter gefassten Begriff Technologieentwicklung vorzieht.\footnote{\citeNP<Vgl.>[S.~36--42.]{Porter1985}}

Als Unterstützung für den Entscheidungsprozess, welche Technologien potentiell wertvoll im Sinne der Wettbewerbsfähigkeit sind und mit welchen Mitteln die Technologieentwicklung stattfinden sollte, wurden bereits in den 1980er-Jahren Konzepte für das Portfoliomanagement von Technologien entwickelt. Sie sollen die begrenzten Ressourcen eines Unternehmens ausgerichtet an Risiken, Erträgen und der Unternehmensstrategie effizient verteilen.\footnote{\citeNP<Vgl.>[S.~383f.]{Jolly2003}}

Die wichtigsten Kriterien für die Auswahl der zu untersuchenden bzw. einzusetzenden Technologien ergeben sich aus der Auswertung beeinflussbarer und unbeeinflussbarer Faktoren. Bei den ersteren handelt es sich um firmenspezifische Merkmale sowie den Umgang mit der Wettbewerbssituation.\footnote{\citeNP<Vgl.>[S.~188.]{Sinha2005}} Die Fachkompetenz der Mitarbeiter kann beispielsweise durch Fortbildungen beeinflusst werden. Ebenso obliegt die Entscheidung über strategische Maßnahmen zur Konkurrenzabwehr in der eigenen Hand. Bei den unbeeinflussbaren Faktoren spielt vor allem die Marktsituation eine Rolle. Zum Beispiel wird in schnell wachsenden Marktsegmenten mit einer Vielzahl an Konkurrenten der Zugzwang Alleinstellungsmerkmale zu schaffen durch den Markt vorgegeben. Auch politische Entscheidungen, die direkt oder indirekt auf das eigene Handeln einwirken, liegen in der Regel nicht im Entscheidungsfreiraum eines Unternehmens.\footnote{\citeNP<Vgl.>[S.~188f.]{Sinha2005}}

Aus der Auswertung dieser Kriterien resultiert eine Tendenz, die genauere Prognosen für den optimalen Zeitpunkt des möglichen Einsatzes einer Technologie zulässt.\footnote{\citeNP<Vgl.>[S.~519.]{Dickinson2001}} Basierend darauf können im Hinblick auf potentielle Risiken und Gewinne Investitionen getätigt werden.\footnote{\citeNP<Vgl.>[S.~804.]{Dolci2014}}

\subsection{Einflussfaktoren auf Technologietrends}
Nach \shortciteauthor{Worlton1988} durchlaufen Technologietrends üblicherweise vier Phasen bis zur Marktreife: Erfindung, Innovation, Verschmelzung und Verbreitung. In der ersten Phase wird neues Wissen oder ein neues technisches Instrument generiert bzw. entwickelt. Die Erfindung allein ist dann allerdings in den meisten Fällen noch nicht marktreif, da der konkrete praktische Nutzen hier noch unbekannt ist. Dieser wird während der Innovationsphase durch Iterationen von Versuch und Irrtum in Ansätzen ermittelt. In Phase drei, der Verschmelzung, folgt erst der Durchbruch, indem verschiedene Technologien mit der neuen Erfindung kombiniert werden und ein neues Produkt bilden. Die vollständige Marktreife wird in Phase vier erreicht, wenn das neue Produkt in Stückzahlen produziert werden kann, die der Nachfrage des Marktes gerecht wird.\footnote{\citeNP<Vgl.>[S.~313.]{Worlton1988}}

In der wissenschaftlichen Literatur gibt es unterschiedliche Theorien über die treibende Kraft der Technologieentwicklung. Dabei sind zwei herrschende Meinungen festzustellen. Die erste besagt, dass Technologietrends durch äußere Einflüsse, wie den Verbraucher oder Anforderungen des Marktes geprägt werden. Demgegenüber lautet die zweite Annahme, dass der Anbieter im Rahmen seiner Kompetenzen und seines Einsatzes von F\&E Technologietrends ausschlaggebend prägt.\footnote{\citeNP<Vgl.>[S.~611.]{Adner2001}}

\shortciteauthor{Adovamicius2008} zufolge ist eine monokausale Betrachtung für die Prognose von Technologietrends nicht zielführend, da sie die Ursachen trivialisiert, die Komplexität bei der Analyse aber nicht reduziert. Sie bevorzugen die Betrachtung von Technologien in einem Ökosystem, wo sie geboren werden, sich von da an stets gegenseitig beeinflussen, folglich einen evolutionären Entwicklungsprozess durchlaufen und schließlich wieder verschwinden können. Um die Komplexität der Technologielandschaft zu verringern, widmen sie sich technischen Korrelationen zwischen verschiedenen Technologien, anstatt die äußeren Einflüsse von Angebot und Nachfrage als alleinigen Impulsgeber zu verstehen.\footnote{\citeNP<Vgl.>[S.~785.]{Adovamicius2008}}

Um das Muster der gegenseitigen Beeinflussung nachzuvollziehen, ist eine Unterteilung in folgende Rollen hilfreich:\footnote{\citeNP<Vgl.>[S.~786.]{Adovamicius2008}}
\begin{description}
	\item[Komponenten:] Teilstücke von Technologien, die kombiniert ein Produkt bilden.
	\item[Produkte:] Zusammengesetzte Komponenten, die mit dem Anwender interagieren.
	\item[Infrastruktur:] Unterstützen und erweitern die Nutzung eines Produktes.
\end{description}

%In Abbildung \ref{fig:influence_path} sind die möglichen Kombinationen dieser Rollen von Einflüssen vorhandener Technologien auf zukünftige Technologien abgebildet.
%
%\begin{figure}
%	\centering
%	\caption{Korrelation der Beeinflussung im technologischen Ökosystem}
%	\includegraphics[width=0.9\linewidth]{pdf/tabelle1}
%	\caption*{\protect\citeNP<Quelle: In Anlehnung an>[S.~785]{Adovamicius2008}}
%	\label{fig:influence_path}
%\end{figure}

In Tabelle \ref{tab:influence_path} sind die möglichen Kombinationen dieser Rollen von Einflüssen vorhandener Technologien auf zukünftige Technologien abgebildet.

\begin{table}
	\caption{Korrelation der Beeinflussung im technologischen Ökosystem}
	\centering
	\label{tab:influence_path}
	\begin{tabularx}{\linewidth}{p{2,3cm}|X|X|X}
         & zukünftige \newline Komponente & zukünftiges \newline Produkt & zukünftige \newline Infrastruktur \\
         \hline
		 vorhandene \newline Komponente & Evolution von Kom\-ponenten & Design \& Verschmelzung & Standards~\& Infra\-struktur\-entwicklung \\
		 \hline
		 vorhandenes \newline Produkt & Produktgetriebene Komponenetenentwicklung & Produktintegration \& -evolution & Verbreitung \& Akzeptanz \\
		 \hline
		 vorhandene \newline Infrastruktur & Infra\-struktur\-getriebene Komponenten\-entwicklung & Infra\-struktur\-fördernde Produkt\-entwicklung & Evolution des Sup\-ports \\
	\end{tabularx}
	\caption*{\protect\citeNP<Quelle: In Anlehnung an>[S.~785]{Adovamicius2008}}
\end{table}

\subsection{Informationsquellen von Unternehmen}
Die globale Technologieentwicklung ist umfangreich und schnelllebig, und dennoch ist sie für Unternehmen von essentieller Bedeutung, da sie im Sinne der Wettbewerbsfähigkeit zugleich Chancen und Risiken birgt. Dies hängt davon ab, ob ein Technologietrend frühzeitig oder aber zu spät erkannt bzw. eingesetzt wird. Das macht die Entscheidung über Investitionen in aufstrebende Technologien zu einer Herausforderung, welche insbesondere durch die Beschaffung von Informationen zu bewältigen ist.\footnote{\citeNP<Vgl.>[S.~780.]{Adovamicius2008}}

Sich dabei maßgeblich auf Informationen von Herstellern dieser Technologien zu verlassen, birgt insofern Risiken, als die Firmeninteressen beider Seiten aufeinandertreffen und nicht zweifelsfrei von einer objektiven Betrachtung der Gegenseite auszugehen ist. Infolgedessen sind Marktforschungsunternehmen entstanden, die Unternehmen bei strategischen Entscheidungen in Fragen der Technologieentwicklung mit ihrer Expertise beratend zur Seite stehen.\footnote{\citeNP<Vgl.>[S.~57.]{Pollock2018}}

Als die größten Marktforschungsunternehmen für Technologien mit einem Marktanteil von zusammen 44~\% im vergangenen Jahr sind Gartner, Forrester und IDC zu nennen, wobei Gartner als einflussreichster und größter Vertreter gilt.\footnote{\citeNP<Vgl.>[o. S.]{Chapple2017}}

Der jährlich herausgegebene \glqq Hype Cycle\grqq~ist erstmalig im Jahre 1995 erschienen und genießt seitdem hohe Anerkennung seitens der praktizierenden Wirtschaft.\footnote{\citeNP<Vgl.>[S.~12.]{Jarvenpaa2008}}

\subsection{Gartner’s Hype Cycle for Emerging Technologies}
Der \glqq Gartner Hype Cycle\grqq~stellt den üblichen Verlauf einer Innovation von einem zu frühen Enthusiasmus über die Desillusionierung bis hin zur Erkenntnis über die mögliche Marktrelevanz dar. Jährlich werden über neunzig verschiedene \glqq Hype Cycles\grqq~ erstellt und publiziert.\footnote{\citeNP<Vgl.>[S.~3.]{Fenn2017}}

Dabei werden, wie bereits in Abbildung \ref{fig:ghc_raw} dargestellt, folgende Phasen durchlaufen:\footnote{\citeNP<Vgl.>[S.~4f.]{Fenn2017}}
\begin{description}
	\item[Innovation Trigger:] Der \glqq Hype Cycle\grqq~startet dort, wo ein Durchbruch bei einer Innovation erreicht und das Interesse der Medien sowie Unternehmen geweckt wurde.
	\item[Peak of Inflated Expectations:] Die nächste Stufe wird erreicht, wenn eine Reihe von Weiterentwicklungen überzogene Erwartungen an die Technologie hegen lassen.
	\item[Trough of Disillusionment:] Die Feststellung, dass die überzogenen Erwartungen nicht der Realität entsprechen, lässt den Enthusiasmus vorerst wieder sinken.
	\item[Slope of Enlightenment:] Nachdem erste Anwender die initialen Startschwierigkeiten überwunden und kommerziellen Nutzen erzielt haben, pendeln sich die Erwartungen mit steigender Tendenz ein.
	\item[Plateau of Productivity:] Dieser Erfolg führt schließlich dazu, dass die Technologie in immer mehr Unternehmen zum Einsatz kommt, da das Risiko einer Fehlinvestition überschaubar wird. Die Kurve steigt ein weiteres Mal an und stabilisiert sich.
\end{description}

Der \glqq Hype Cycle for Emerging Technologies\grqq~stellt eine Selektion aus etwa \numprint{2000} wichtigsten, aufstrebenden Technologien zusammen, die voraussichtlich innerhalb der nächsten fünf bis zehn Jahre ein hohes Maß an Wettbewerbsfähigkeit generieren werden.\footnote{\citeNP<Vgl.>[S.~3.]{ghc2017}}

Dabei ist zu berücksichtigen, dass eine Technologie nicht zwangsläufig alle Phasen des \glqq Hype Cycle\grqq~durchlaufen muss. Viele Trendtechnologien erreichen das \glqq Plateau of Productivity\grqq~erst gar nicht. Sie können auch beispielsweise erstmalig im \glqq Peak of Inflated Expectations~erscheinen, wenn sie etwa zum Zeitpunkt der vorangegangenen Ausgabe nicht genügend im Fokus der relevanten Interessengruppe standen, innerhalb des jährlichen Veröffentlichungszyklus jedoch stark an Bedeutung hinzugewannen.\footnote{\citeNP<Vgl.>[S.~10f.]{Fenn2017}} Andersherum kann eine Technologie, die bereits in den ersten beiden Stufen der Aufschwungphase erschienen ist, in einer nächsten Ausgabe für immer herausfallen oder aber in einer der folgenden Ausgaben erneut erscheinen. Wichtig ist zu wissen, dass alle abgebildeten Technologien eine gewisse Schwelle an Aufmerksamkeit überschritten haben und sich durch Abstufungen in Reife und Intensität der Erwartungen unterscheiden.

Da die Phase des \glqq Peak of Inflated Expectations\grqq~ den Zeitpunkt mit den höchsten Erwartungen anzeigt, dient sie für die vorliegende Arbeit als Indikator für Technologietrends.

\subsection{Schnittstellen mit der akademischen Forschung}
Wie in Abschnitt \ref{sec:sight} bereits erwähnt, ist die Erlangung von Wissen für Unternehmen von großer Bedeutung, wenn es kommerzialisiert werden kann. Universitäten spielen dabei eine zentrale Rolle, indem sie unter anderem durch Ausbildung von potentiellen Mitarbeitern und kollaborativer Forschung mit Unternehmen zur Kommerzialisierung dieses Wissens beitragen.\footnote{\citeNP<Vgl.>[S.~564.]{Boehm2013}}

Für Unternehmen ist vor allem Wissen in Form von Technologien relevant. In Zeiten knapper Ressourcen sowie Forderungen nach immer kürzerem Time-to-Market nehmen Unternehmen vermehrt Gelegenheiten zur Zusammenarbeit wahr.\footnote{\citeNP<Vgl.>[S.~86.]{Calvert2003}}

Eine weitere Schnittstelle stellen aus Universitäten ausgegründete Unternehmen wie etwa Google dar, die eine akademische Innovation kommerzialisieren und weiterentwickeln.\footnote{\citeNP<Vgl.>[S.~274.]{Shane2015}}

In beiden Fällen entstehen Gemeinsamkeiten beim Verständnis für Technologietrends, die abhängig von der jeweiligen Interessenlage auch voneinander wieder divergieren können.