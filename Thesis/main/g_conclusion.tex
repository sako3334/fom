\section{Fazit}
Neue Technologien haben weitreichenden Einfluss und Konsequenzen auf das ge\-samt\-ge\-sellschaftliche Leben. Es besteht daher ein Interesse seitens Unternehmer wie auch Forscher, einen Technologietrend im Voraus zu erkennen, da es ihnen Wettbewerbsvorteile bietet.\footnote{\citeNP<Vgl.>[S.~114.]{Dotsika2017}}

In vorliegender Arbeit wurde in diesem Sinne ein Versuch unternommen, die Entstehung eines Trends einer bestimmten Domäne zuzuordnen, um eine mögliche Quelle für eine Vorhersage zu bestimmen. Für die Untersuchung dienten dabei aus Sicht der Wirtschaft als \glqq Emerging Technologies\grqq~geltende Technologien, die sich per Definition in der praktischen Anwendung noch nicht etabliert haben.\footnote{\citeNP<Vgl.>[S.~285.]{Li2018}}

Für die ausgewählte Datenbasis konnte dabei kein eindeutiges Muster für eine Beeinflussung aus einer bestimmten Richtungen festgestellt werden. Die Veröffentlichungszahlen in der wissenschaftlichen Literatur entsprechen nicht dem Verhältnis der Positionen im \glqq Hype Cycle\grqq. Auch weichen die Bezeichnungen der Technologien in vielen Fällen voneinander ab. Ein möglicher Grund dafür ist die Auswahl des Technologiebegriffes durch das Unternehmen \glqq Gartner\grqq, welche die dafür notwendigen Informationen aus den befragten Partnerunternehmen sammelt und sich nach Auswertung für eine Benennung entscheidet. Diese stimmen unter Umständen nicht mit der wissenschaftlichen Benennung überein, da sie unter anderem dem Zwecke des Marketing dienen.

Deshalb wurden die analysierten Technologien nach unterschiedlichem Kriterien klassifiziert. In dieser Betrachtung fiel auf, dass aus den vorliegenden Technologien die gesellschaftlich bedeutungsvollsten, wie etwa \glqq Machine Learning\grqq~oder \glqq Autonomous Vehicles\grqq, die größte Relevanz in der Wissenschaft haben. Folglich gibt es auch Gemeinsamkeiten beim Trendverständnis, die sich bspw. in der Anhäufung von Technologien auf Basis der künstlichen Intelligenz widerspiegeln. Technologien, die für Problemlösungen in einem engeren Kontext dienen, wie \glqq Micro Data Centers\grqq~oder \glqq Software-Defined Anything\grqq, genossen kein so großes Interesse. 

Die Auswertung der Suchergebnisse getrennt nach Literaturdatenbank und Veröffentlichungsform ergab, dass Fachartikel im \ac{WoS} sowie Konferenzbeiträge im \ac{IEEE} in der Regel die meisten Ergebnisse zu einem Thema liefern und oft ähnliche Mengen an Veröffentlichungen aufweisen. Die Datenerhebung hätte sich auf diese beiden Typen beschränken können.

Die Analyse ergibt, dass die wissenschaftliche Forschung vorwiegend theoretische Grundlagen einer Technologie ergründet. Die Wirtschaft übernimmt diese ab einem bestimmten Reifegrad, um sie bis zur Praktikabilität weiterzuentwickeln. Die Rollen sind somit klar abgegrenzt. Die Vermutung der vorhergehenden Forschung auf Diskrepanzen kann somit bestätigt werden. Die Bestimmung von globalen Trendthemen in der Wissenschaft kann in weiterer Forschung analysiert werden, da sie hier lediglich in Relation zur Wirtschaft betrachtet wurde. Dazu wäre allerdings eine umfangreichere Methodik bspw. durch \glqq Text-Mining-Verfahren\grqq~der Stichwörter notwendig, um das tatsächliche Thema der Exportdaten massenhaft festzustellen.