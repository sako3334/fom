\section{Fazit}
Neue Technologien haben weitreichenden Einfluss und Konsequenzen auf das ge\-samt\-ge\-sellschaftliche Leben. Es besteht daher ein Interesse seitens Unternehmer wie auch Forscher, einen Technologietrend im Voraus zu erkennen, da es ihnen Wettbewerbsvorteile bietet.\footnote{\citeNP<Vgl.>[S.~114.]{Dotsika2017}}

In vorliegender Arbeit wurde in diesem Sinne ein Versuch unternommen, die Entstehung eines Trends einer bestimmten Domäne zuzuordnen, um eine mögliche Quelle für eine Vorhersage zu bestimmen. Für die Untersuchung dienten dabei aus Sicht der Wirtschaft als \glqq Emerging Technologies\grqq~geltende Technologien, die sich per Definition in der praktischen Anwendung noch nicht etabliert haben.\footnote{\citeNP<Vgl.>[S.~285.]{Li2018}}

Für die ausgewählte Datenbasis konnte dabei kein eindeutiges Muster für eine Beeinflussung aus einer bestimmten Richtungen festgestellt werden. Die Veröffentlichungszahlen in der wissenschaftlichen Literatur entsprechen nicht dem Verhältnis der Positionen im \glqq Hype Cycle\grqq. Auch weichen die Bezeichnungen der Technologien in vielen Fällen voneinander ab. Ein möglicher Grund dafür ist die Auswahl des Technologiebegriffes durch das Unternehmen \glqq Gartner\grqq, welche die dafür notwendigen Informationen aus den befragten Partnerunternehmen sammelt und sich nach Auswertung für eine Benennung entscheidet. Diese stimmen unter Umständen nicht mit der wissenschaftlichen Benennung überein, da sie unter anderem dem Zwecke des Marketing dienen.

Deshalb wurden die analysierten Technologien mit unterschiedlichem Schwerpunkt klassifiziert. In dieser Betrachtung fiel auf, dass aus den vorliegenden Technologien die gesellschaftlich bedeutungsvollsten, wie etwa \glqq Machine Learning\grqq~oder \glqq Autonomous Vehicles\grqq, die größte Relevanz in der Wissenschaft haben. Technologien, die für auf Problemlösungen in engerem Kontext dienen, wie \glqq Micro Data Centers\grqq~oder \glqq Software-Defined Anything\grqq, genießen kein so großes Interesse.
Dennoch gibt es auch Gemeinsamkeiten beim Trendverständnis, die sich bspw. in der Anhäufung von Technologien auf Basis der künstlichen Intelligenz widerspiegeln. 

Klare Rollenverteilung: Es scheint als wenn die theoretischen Grundlagen in der Wissenschaft entwickelt wird und von der Wirtschaft bis zur Praktikabilität weiterentwickelt wird. Es egth auch andersherum blockchain


bisherige forsvchun wenig veraltet

%Grundlage für weitere Forschung
%welche synonyme beim ghc zum einsatz kommen nicht bekannt. evtl um einzigartigkeit zu betonen <- marketing
%
%theoretischer rahmen vll in uni aber trend erst durch vermarktung eines produktes
%
%Unterschiedliche Bezeichnung, unterschiedliche Mengen, vergleichbare Kurvenverläufe
%
%Auffälligkeiten verschwimmen über die Betrachtung der einzelnen V-Typen